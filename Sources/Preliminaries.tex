%%%%%%%%%%%%%%%%%%%%%%%%%%%%%%%%%%%%%%%%%%%%%%%%%%%%%%%%%%%%%%%%%%%%%%%%
\chapter{Preliminaries}
%%%%%%%%%%%%%%%%%%%%%%%%%%%%%%%%%%%%%%%%%%%%%%%%%%%%%%%%%%%%%%%%%%%%%%%%
This chapter fixes notation and collects the background material used throughout the thesis. The central objects are \'{e}tale groupoids, a common framework that encompasses discrete groups, equivalence relations, and dynamical systems. We begin with the algebraic notion of a groupoid and its orbit decomposition. We then introduce isotropy and principality, which measure stabiliser phenomena. Next we pass to topological groupoids, where all structure maps are continuous. Finally we specialise to \'{e}tale groupoids, characterised by the source map being a local homeomorphism. In this case the range map is also a local homeomorphism since \(r=s\circ i\). This regularity is the input for the analytic and homological constructions in l
\begin{setting}
Throughout we write arrows in groupoids as \(\gamma,\eta\) and units as \(x,y,z\).
\end{setting}

ater chapters.

\section{Groupoids}
A groupoid is a small category in which every morphism is invertible. Concretely, it is a set \(\G\) of arrows with partially defined composition \(m:\G_2\to\G\), where
\[
\G_2 \coloneqq \{(\gamma,\eta)\in \G\times \G \mid s(\gamma)=r(\eta)\},
\]
together with an object set, the unit space \(\G_0\), range and source maps \(r,s:\G\to\G_0\), and inversion \(\gamma\mapsto \gamma^{-1}\), such that composition is associative on \(\G_2\) and units and inverses satisfy the usual identities \cite[Remark~2.1.5]{Sims2018}. Groupoids are a robust algebraic model for orbit and quotient spaces that can fail to be well behaved topologically. Typical examples are non-\(T_1\) quotients such as the shift space modulo the shift, or the circle modulo an irrational rotation \cite[p.~1]{Sims2018}. As an illustration, let \(\mathbb{T}=\mathbb{R}/\mathbb{Z}\) and fix \(\alpha\in\mathbb{R}\setminus\mathbb{Q}\). Define \(r_\alpha:\mathbb{T}\to\mathbb{T}\) by \(r_\alpha(x)=x+\alpha \bmod 1\), and write \(x\sim y\) if there exists \(n\in\mathbb{Z}\) with \(y=r_\alpha^{n}(x)\). Then \(\mathbb{T}/\!\sim\) is not \(T_1\). Indeed, by Dirichlet’s Approximation Theorem \cite[Theorem~185]{HardyWright2008}, for every interval \(I\subset\mathbb{T}\) there exists \(n\in\mathbb{Z}\) with \(n\alpha \bmod 1\in I\). Hence each orbit \(\{x+n\alpha \bmod 1 \mid n\in\mathbb{Z}\}\) is dense and therefore not closed in \(\mathbb{T}\). Let \(\pi:\mathbb{T}\to \mathbb{T}/\!\sim\) be the quotient map. Then \(\pi^{-1}([x])\) is the non-closed orbit of \(x\). If \([x]\) were closed in \(\mathbb{T}/\!\sim\), its preimage \(\pi^{-1}([x])\) would be closed in \(\mathbb{T}\), a contradiction. We include the proof for completeness.

\begin{setting}
For \(x\in\mathbb{R}\), write its fractional part as \(\{x\}\coloneqq x-\lfloor x\rfloor\in[0,1)\), where \(\lfloor x\rfloor\coloneqq \max\{n\in\mathbb{Z}\mid n\le x\}\). The distance to the nearest integer is \(\|x\|\coloneqq \min\bigl\{\{x\}, 1-\{x\}\bigr\}\in\bigl[0,\tfrac12\bigr]\).
\end{setting}

\begin{theorem}[Dirichlet Approximation Theorem {\cite[Theorem~185]{HardyWright2008}}]
\label{DirichletApproximation}
For any real \(\alpha\) and any integer \(Q\ge 1\), there exist integers \(p,q\) with \(1\le q\le Q\) such that
\[
\lvert q\alpha-p\rvert\le \tfrac{1}{Q},\quad \text{that is}\quad \|q\alpha\|\le \tfrac{1}{Q}.
\]
\end{theorem}

\begin{proof}
Consider the \(Q{+}1\) numbers \(\{0\},\{\alpha\},\dots,\{Q\alpha\}\) in \([0,1)\).
Partition \([0,1)\) into \(Q\) half-open intervals \(I_j=[\tfrac{j}{Q},\tfrac{j+1}{Q})\) for \(j=0,1,\dots,Q-1\).
By the pigeonhole principle, two of the \(Q{+}1\) fractional parts, say \(\{\ell\alpha\}\) and \(\{k\alpha\}\) with \(0\le k<\ell\le Q\), lie in the same \(I_j\).
Hence
\[
d\coloneqq\bigl|\{\ell\alpha\}-\{k\alpha\}\bigr|<\frac{1}{Q}.
\]
Let \(q=\ell-k\) with \(1\le q\le Q\) and set \(p=\lfloor\ell\alpha\rfloor-\lfloor k\alpha\rfloor\).
Then
\[
q\alpha-p=(\ell\alpha-\lfloor\ell\alpha\rfloor)-(k\alpha-\lfloor k\alpha\rfloor)=\{\ell\alpha\}-\{k\alpha\},
\]
so \(|q\alpha-p|<1/Q\).
In particular \(\|q\alpha\|\le |q\alpha-p|<1/Q\). Since \(Q\) is arbitrary, for every \(\varepsilon>0\) there exists \(q\in\mathbb{N}\) with \(\|q\alpha\|<\varepsilon\), hence the set of such \(q\) is infinite. If \(\alpha\) is rational, then \(\|q\alpha\|=0\) for infinitely many \(q\). If \(\alpha\) is irrational, continued fractions yield infinitely many coprime \(p,q\) with \(|q\alpha-p|<1/q^{2}\), hence \(\|q\alpha\|<1/q\) for infinitely many \(q\) \cite[Chapter~X]{HardyWright2008}.
\end{proof}

\begin{example}~
\label{ex:orbits}
\begin{itemize}
\item \textbf{Irrational rotation on the circle.} Let \(\mathbb{T}=\mathbb{R}/\mathbb{Z}\) with addition modulo \(1\) and \(r_\alpha:\mathbb{T}\to\mathbb{T}\), \(r_\alpha(x)=x+\alpha \bmod 1\), for \(\alpha\in\mathbb{R}\setminus\mathbb{Q}\). Consider the orbit relation \(x\sim y \Leftrightarrow \exists n\in\mathbb{Z}: y=r_\alpha^{n}(x)\) and the quotient map \(\pi:\mathbb{T}\to\mathbb{T}/{\sim}\). For irrational \(\alpha\), each orbit \(O(x)=\{r_\alpha^{n}(x)\mid n\in\mathbb{Z}\}\) is dense in \(\mathbb{T}\): let \(I\subset[0,1)\) be an interval of length \(l>0\) and write \(I=[a,a+l)\bmod 1\). Apply the pigeonhole argument used in Dirichlet’s theorem to the \(Q{+}1\) numbers \(\{x-a\},\ \{x+\alpha-a\},\ \dots,\ \{x+Q\alpha-a\}\in[0,1)\) and to the partition of \([0,1)\) into \(Q\) half-open intervals of length \(1/Q\). Then there exist \(0\le k<\ell\le Q\) with \(\{x+\ell\alpha-a\}\) and \(\{x+k\alpha-a\}\) lying in the same subinterval. Without loss of generality assume \(\{x+\ell\alpha-a\}\ge \{x+k\alpha-a\}\). Hence
\[
0\le \{x+\ell\alpha-a\}-\{x+k\alpha-a\}<\frac{1}{Q}.
\]
Set \(q\coloneqq \ell-k\) and \(p\coloneqq \lfloor x+\ell\alpha-a\rfloor-\lfloor x+k\alpha-a\rfloor\). Then
\[
\{x+q\alpha-a\}
=\{(x+\ell\alpha-a)-(x+k\alpha-a)\}
=\{x+\ell\alpha-a\}-\{x+k\alpha-a\}
<\frac{1}{Q}<l,
\]
so \(x+q\alpha \bmod 1\in I\). As \(I\) was arbitrary, \(O(x)\) is dense and not closed. Since \(\pi\) is continuous and \(\pi^{-1}([x])=O(x)\) is not closed, the singleton \(\{[x]\}\) is not closed in \(\mathbb{T}/{\sim}\), thus not \(T_1\).

\begin{remark}
Dirichlet provides a small step \(q\) with \(\delta\coloneqq\|q\alpha\|<l\). Stepping by \(\pm q\alpha\), that is along the subsequence \(r_\alpha^{n_0\pm m q}(x)\), forces a hit in any arc of length \(>\delta\).
\end{remark}

\item \textbf{Shift space modulo the shift.} An alphabet is a finite, nonempty set \(A\) with the discrete topology. The one-sided full shift space over \(A\) is \(\Sigma\coloneqq A^{\mathbb{N}}\) with the product topology. A basic open set, also cylinder, determined by a word \(w=(w_0,\dots,w_{k-1})\in A^k\) for \(k \in \mathbb{N}\) is
\[
[w]\coloneqq\{x\in\Sigma \mid x_0=w_0,\dots,x_{k-1}=w_{k-1}\}.
\]
The shift map \(\sigma:\Sigma\to\Sigma\), \((\sigma(x))_n\coloneqq x_{n+1}\), is continuous and surjective. Consider the equivalence relation \(x\sim y\Leftrightarrow O^+(x)=O^+(y)\), where \(O^+(x)\coloneqq\{\sigma^n(x)\mid n\in\mathbb{N}\}\). Let \(\pi_\Sigma:\Sigma\to\Sigma/{\sim}\) be the quotient map, so \(\pi_\Sigma^{-1}([x])=O^+(x)\) for each \(x\). Fix distinct \(a,b\in A\) and set \(z\coloneqq(b)_{n \in \mathbb{N}}\in\Sigma\). Define \(x\in\Sigma\) by placing \(a\) at positions \(n_j\coloneqq 2^j\) and \(b\) elsewhere, that is \(x_{n_j}=a\) for \(j\ge 1\) and \(x_n=b\) if \(n\notin \{2^j \mid j\ge 1\}\). Then \(O^+(x)=\{\sigma^n(x)\mid n\in\mathbb{N}\}\) is not closed and \(z\in\overline{O^+(x)}\setminus O^+(x)\). Let \(U\) be any cylinder neighbourhood of \(z\) determined by its first \(k\) coordinates, so \(U=[w]\) with \(w=b^k\). Choose \(j\) so large that \(2^{j-1}>k+1\) and set \(n\coloneqq 2^j-(k+1)\). The symbol \(a\) at position \(2^j\) moves to position \(k+1\) in \(\sigma^n(x)\), hence the first \(k\) coordinates of \(\sigma^n(x)\) are all \(b\). Moreover, any earlier \(a\) at position \(2^m\) with \(m\le j-1\) moves to position \(2^m-n\le (k+1)-2^{j-1}<0\), so it does not affect these first \(k\) coordinates. Thus \(\sigma^n(x)\in U\). As \(k\) was arbitrary, \(z\in\overline{O^+(x)}\). On the other hand, \(z\notin O^+(x)\): if \(\sigma^n(x)=z\) for some \(n\), then \(x_{n+i}=b\) for all \(i\ge 0\), contradicting the infinitely many occurrences \(x_{2^j}=a\). Therefore \(O^+(x)\) is not closed in \(\Sigma\). Since \(\pi_\Sigma^{-1}([\pi_\Sigma(x)])=O^+(x)\) is not closed and \(\pi_\Sigma\) is continuous, the singleton \(\{\pi_\Sigma(x)\}\) is not closed in \(\Sigma/{\sim}\). Hence \(\Sigma/{\sim}\) is not \(T_1\).
\end{itemize}
\end{example}

To connect the examples above with the general theory, we begin with a hands-on axiomatisation of a groupoid in the notation \((\mathcal{G},\mathcal{G}_0,r,s,m,i)\) and with left--to--right composition \(\gamma\cdot\eta\), defined when \(s(\gamma)=r(\eta)\). This presentation is slightly redundant but keeps all structure maps visible and is useful for intuition and for later constructions, such as nerves and pushforwards. After deriving a few basic consequences, we recast the same object in its concise categorical form, namely a small category in which every arrow is invertible, and verify that the two formulations are equivalent. For \'{e}tale groupoids we record the equivalent reformulation that \(r\) and \(s\) are local homeomorphisms, aligning with the simplicial setup in Chapter~\ref{chap:moore-homology}.

\begin{definition}[Groupoid {\cite[Remark~2.1.5]{Sims2018}}]
\label{def:groupoid}
A groupoid is a sextuple \((\mathcal{G},\mathcal{G}_0,r,s,m,i)\) consisting of a set of morphisms \(\mathcal{G}\), a distinguished subset \(\mathcal{G}_0\subseteq\mathcal{G}\) of units, range and source maps \(r,s:\mathcal{G}\to\mathcal{G}_0\), a partially defined multiplication on composable pairs
\[
\cdot: \mathcal{G}_2\coloneqq\{(\gamma,\eta)\in\mathcal{G}\times\mathcal{G}\mid s(\gamma)=r(\eta)\}\ \to\ \mathcal{G},\quad
(\gamma,\eta)\mapsto \gamma\cdot\eta,
\]
and an inversion map \(i:\mathcal{G}\to\mathcal{G}\), \(i(\gamma)\coloneqq \gamma^{-1}\), subject to:
\begin{enumerate}[noitemsep,nolistsep]
\item[(G1)] \(r(x)=x=s(x)\) for all \(x\in\mathcal{G}_0\).
\item[(G2)] \(r(\gamma)\cdot\gamma=\gamma=\gamma\cdot s(\gamma)\) for all \(\gamma\in\mathcal{G}\).
\item[(G3)] \(r(\gamma^{-1})=s(\gamma)\) and \(s(\gamma^{-1})=r(\gamma)\) for all \(\gamma\in\mathcal{G}\).
\item[(G4)] \(\gamma^{-1}\cdot\gamma=s(\gamma)\) and \(\gamma\cdot\gamma^{-1}=r(\gamma)\) for all \(\gamma\in\mathcal{G}\).
\item[(G5)] If \(s(\alpha)=r(\beta)\), then \(r(\alpha\cdot\beta)=r(\alpha)\) and \(s(\alpha\cdot\beta)=s(\beta)\).
\item[(G6)] If \(s(\alpha)=r(\beta)\) and \(s(\beta)=r(\gamma)\), then \((\alpha\cdot\beta)\cdot\gamma=\alpha\cdot(\beta\cdot\gamma)\).
\end{enumerate}
\end{definition}

\begin{remark}
The unit space is
\[
\mathcal{G}_{0} \coloneqq \{\gamma^{-1}\gamma \mid \gamma\in\mathcal{G}\}
= \{\gamma\gamma^{-1} \mid \gamma\in\mathcal{G}\} \subseteq \mathcal{G},
\]
and its elements are the unit arrows. Concretely, for \(x\in\mathcal{G}_{0}\) one has \(x\gamma=\gamma\) whenever \(r(\gamma)=x\) and \(\gamma x=\gamma\) whenever \(s(\gamma)=x\). The source and range maps \(s,r:\mathcal{G}\to\mathcal{G}_{0}\) decompose \(\mathcal{G}\) into the disjoint unions \(\mathcal{G}=\bigsqcup_{x\in\mathcal{G}_{0}} s^{-1}(x)=\bigsqcup_{x\in\mathcal{G}_{0}} r^{-1}(x)\). The isotropy group at \(x\in\mathcal{G}_{0}\) is
\[
\mathcal{G}_x \coloneqq \{\gamma\in\mathcal{G} \mid s(\gamma)=r(\gamma)=x\},
\]
a group under multiplication. If \(\mathcal{G}\) is a topological groupoid, then \(\mathcal{G}_{0}\), each fibre \(s^{-1}(x)\) and \(r^{-1}(x)\), and each \(\mathcal{G}_x\) inherit the subspace topology. In the special case of a group, a one-object groupoid, \(\mathcal{G}_{0}=\{e\}\) and \(\mathcal{G}_e=\mathcal{G}\).

In general, we will write \(\G\) for the arrow space and denote the groupoid by the sextuple or quadruple as in Definition~\ref{def:groupoid} or Definition~\ref{def:groupoid2}. However, sometimes it might be convenient for notational reasons to denote the objects as usual by \(\G_0\) and the morphisms by \(\G_1\), see Example~\ref{ex:etale-groupoids}.
\end{remark}

\begin{remark}
Writing fibre products over \(\mathcal{G}_{0}\) with respect to \(s\) and \(r\), we have
\[
\mathcal{G}_n = \{(\gamma_1,\dots,\gamma_n)\in\mathcal{G}^{n} \mid s(\gamma_i)=r(\gamma_{i+1}) \text{ for } 1\le i<n\} \cong
\underbrace{\mathcal{G}\stimesr\mathcal{G}\stimesr\cdots\stimesr\mathcal{G}}_{n\ \text{factors}},
\]
where \(\mathcal{G}\stimesr\mathcal{G}=\{(\gamma,\eta)\in\mathcal{G}\times\mathcal{G} \mid s(\gamma)=r(\eta)\}\). For \(n=1\) set \(\mathcal{G}_1=\mathcal{G}\). We equip \(\mathcal{G}_n\) with the subspace topology inherited from \(\mathcal{G}^{n}\).
\end{remark}

\begin{definition}
Let \(\bigl(\mathcal G,\mathcal G_{0},r_{\mathcal G},s_{\mathcal G},\cdot_{\mathcal G},-_{\mathcal G}^{-1}\bigr)\) and
\(\bigl(\mathcal H,\mathcal H_{0},r_{\mathcal H},s_{\mathcal H},\cdot_{\mathcal H},-_{\mathcal H}^{-1}\bigr)\) be two groupoids. A homomorphism of groupoids is a functor consisting of a map \(F:\mathcal G\to\mathcal H\) and a map
\(F_{0}:\mathcal G_{0}\to\mathcal H_{0}\) such that:
\begin{enumerate}[noitemsep,nolistsep]
\item[(F1)] \(F(\mathcal G_{0})\subseteq \mathcal H_{0}\) and \(F|_{\mathcal G_{0}}=F_{0}\).
\item[(F2)] For all \(\gamma\in\mathcal G\),
\(r_{\mathcal H}\bigl(F(\gamma)\bigr)=F_{0}\bigl(r_{\mathcal G}(\gamma)\bigr)\) and
\(s_{\mathcal H}\bigl(F(\gamma)\bigr)=F_{0}\bigl(s_{\mathcal G}(\gamma)\bigr)\).
\item[(F3)] For all \(\alpha,\beta\in\mathcal G\) with \(s_{\mathcal G}(\alpha)=r_{\mathcal G}(\beta)\), \(F(\alpha)\cdot F(\beta)\) is defined and \(F(\alpha\cdot \beta)=F(\alpha)\cdot F(\beta)\).
\end{enumerate}
In particular, \(F(1_x)=1_{F_{0}(x)}\) for all \(x\in\mathcal G_{0}\) and
\(F(\gamma^{-1})=F(\gamma)^{-1}\) for all \(\gamma\in\mathcal G\).
\end{definition}

\begin{remark}
We will write \(\G\) for the data \((\mathcal{G},\mathcal{G}_0,r,s,m,i)\) whenever the unit space, the range and source maps, the multiplication, and the inversion are clear from context.
\end{remark}

\begin{definition}
A homomorphism of groupoids \(F:\G\to\Hh\) is an isomorphism of groupoids if there exists a homomorphism \(G:\Hh\to\G\) with \(G\circ F=\mathrm{id}_{\mathcal{G}}\) and \(F\circ G=\mathrm{id}_{\mathcal{H}}\). If \(F\) admits an inverse homomorphism, then \(F\) is bijective and satisfies (F1)–(F3). Conversely, if \(F\) is bijective and satisfies (F1)–(F3), then the set theoretic inverse \(F^{-1}\) is again a homomorphism and hence \(F\) is an isomorphism of groupoids.
\end{definition}

\begin{remark}
For \'{e}tale groupoids we will require \(F\) and \(F_0\) to be continuous. Since \(\mathcal{G}_0\subseteq\mathcal{G}\) carries the subspace topology, continuity of \(F\) implies continuity of \(F_0=F|_{\mathcal{G}_0}\).
\end{remark}

Principal groupoids capture pure orbit structure without isotropy groups of units, also known as internal stabilisers. Every morphism is determined uniquely by its source and range. For transformation groupoids this corresponds to free actions.

\begin{definition}
We call \(\mathcal{G}\) principal if any of the following equivalent conditions hold:
\begin{enumerate}[noitemsep,nolistsep]
\item[(P1)] \textbf{Trivial isotropy:} \(\forall x\in\mathcal{G}_0\), the isotropy group \(\mathcal{G}_x\coloneqq\{\gamma\in\mathcal{G}\mid r(\gamma)=s(\gamma)=x\}\) equals \(\{x\}\).
\item[(P2)] \textbf{Injective anchor:} \((r,s):\mathcal{G}\to\mathcal{G}_0\times\mathcal{G}_0\) is injective, hence there is at most one arrow \(y\to x\).
\end{enumerate}
\end{definition}

\begin{definition}
Let a discrete group \(\Gamma\) act on a locally compact Hausdorff space \(X\) by homeomorphisms. The action is free if the stabiliser at \(x\) given by \(\Gamma_x\coloneqq\{g\in\Gamma \mid g\cdot x=x\}\) is trivial for every \(x\in X\).
\end{definition}

\begin{proposition}
The transformation groupoid from Example~\ref{ex:groupoids-canon} \(\Gamma\ltimes X\) with objects \(X\), morphisms \(\Gamma\times X\), \(r(g,x)=g\cdot x\), \(s(g,x)=x\), is principal if and only if the action is free.
\end{proposition}

\begin{remark}
Here, \(\Gamma\ltimes X\) is also \'{e}tale, see Section~\ref{chapter:etalegroupoids}. \'{E}taleness does not depend on freeness.
\end{remark}

\begin{proof}
For \(x\in X\), the isotropy group of \(\Gamma\ltimes X\) at \(x\) is \((\Gamma\ltimes X)_x=\{(g,x)\mid g\cdot x=x\}\), which is isomorphic to the stabiliser \(\Gamma_x=\{g\in\Gamma \mid g\cdot x=x\}\) via \((g,x)\mapsto g\). Thus \(\Gamma\ltimes X\) is principal if and only if every \(\Gamma_x\) is trivial, that is the action is free. For \'{e}taleness, assume \(\Gamma\) is discrete and the action is continuous. Give \(\Gamma\times X\) the product topology. For any \((g,x)\in\Gamma\times X\) and any open \(U\ni x\), the restrictions \(s|_{\{g\}\times U}:\{g\}\times U\to U\) and \(r|_{\{g\}\times U}:\{g\}\times U\to g\cdot U\) are homeomorphisms. Hence \(s\) and \(r\) are local homeomorphisms and \(\Gamma\ltimes X\) is \'{e}tale.
\end{proof}

Minimality singles out groupoids with no proper invariant pieces of the unit space. Every orbit is dense, so the dynamics are indecomposable. For transformation groupoids this corresponds to minimal actions.

\begin{definition}[Minimal groupoid]
Let \(\G\) be a topological groupoid. For \(x\in\mathcal{G}_0\) write the orbit \(O(x)\coloneqq \{r(\gamma) \mid \gamma\in\mathcal{G},\ s(\gamma)=x\}\). A subset \(U\subseteq\mathcal{G}_0\) is invariant, also known as saturated, if it is a union of orbits. This means that \(r\bigl(s^{-1}(U)\bigr)=U\). The same condition can be written as \(s\bigl(r^{-1}(U)\bigr)=U\). We call \(\mathcal{G}\) minimal if any of the following hold:
\begin{enumerate}[noitemsep]
\item[(M1)] \textbf{Dense orbits:} \(\overline{O(x)}=\mathcal{G}_0\) for all \(x\in\mathcal{G}_0\).
\item[(M2)] There is no nonempty proper open invariant \(U\subset\mathcal{G}_0\).
\end{enumerate}
\end{definition}

\begin{remark}
For a left action \(\Gamma\rhd X\), the transformation groupoid \(\Gamma\ltimes X\) from Example~\ref{ex:groupoids-canon} is minimal if and only if the action is minimal, hence every \(\Gamma\)-orbit is dense in \(X\).
\end{remark}

\begin{example}[Groupoids {\cite[Ex.~2.1.7–2.1.11]{Sims2018}}]\label{ex:groupoids-canon}~
\begin{itemize}[noitemsep]
\item \textbf{Groups.} Every group \(G\) is a discrete groupoid with one unit:
\(\mathcal{G}=G\), \(\mathcal{G}_0=\{e\}\), \(r(g)=s(g)=e\), multiplication \(g\cdot h=gh\), inversion \(\gamma\mapsto \gamma^{-1}\).

A groupoid is a group if and only if \(\mathcal{G}_0\) is a singleton.

\item \textbf{Group bundles.} Let \(X\) be a set and \(\{G_x\}_{x\in X}\) a family of groups. Put \(\mathcal{G}\coloneqq \bigsqcup_{x\in X}\{x\}\times G_x\), \(\mathcal{G}_0=\{ (x,e_{G_x}) \mid x\in X \}\cong X\), \(r(x,g)=x=s(x,g)\), \((x,g)\cdot(x,h)=(x,gh)\), \((x,g)^{-1}=(x,g^{-1})\). If each \(G_x\) is a topological group and the disjoint union carries the sum topology, this is a topological groupoid, see Definition~\ref{def:topological-groupoid}.

\item \textbf{Equivalence-relation groupoid.} For a set \(X\) and an equivalence relation \(R\subseteq X\times X\), let \(\mathcal{G}=R\), \(\mathcal{G}_0=X\), \(r(x,y)=x\), \(s(x,y)=y\), \((x,y)^{-1}=(y,x)\), and \((x,y)\cdot(y,z)=(x,z)\). If \(X\) is a topological space and \(R\subseteq X\times X\) is equipped with the subspace topology, this is a topological groupoid. It is principal. It is \'{e}tale whenever the coordinate projections are local homeomorphisms, for instance if \(X\) is discrete, or if \(R\) is an \'{e}tale equivalence relation on a Cantor set.

\item \textbf{Matrix finite pair groupoid.} For \(X=\{1,\dots,n\}\) the pair groupoid \(\mathcal{G}=X\times X\) with morphisms \((i,j)\), units \(X\), and composition \((i,k)\cdot(k,j)=(i,j)\) is discrete, principal, and \'{e}tale. Its groupoid \(C^\ast\)-algebra is \(\operatorname{Mat}(n \times n,\mathbb{C})\) with matrix units \(E_{ij}\) that correspond to \((i,j)\).

\item \textbf{Transformation groupoid.} Let a discrete group \(\Gamma\) act on an \(\mathbf{LCH}\) space \(X\) by homeomorphisms. The transformation groupoid \(\mathcal{G}=\Gamma\ltimes X\) has objects \(\mathcal{G}_0=X\), morphisms \(\Gamma\times X\), \(r(g,x)=g\cdot x\), \(s(g,x)=x\), \((g,x)^{-1}=(g^{-1},g\cdot x)\), and \((h,g\cdot x)\cdot(g,x)=(hg,x)\) whenever \(s(h,g\cdot x)=r(g,x)\). Since \(\Gamma\) is discrete, \(r\) and \(s\) are local homeomorphisms. Thus \(\Gamma\ltimes X\) is \'{e}tale. For \(r_\alpha\) on \(\mathbb{T}\), the groupoid \(\mathbb{Z}\ltimes\mathbb{T}\) is Hausdorff, \'{e}tale, principal, and minimal.

\item \textbf{Deaconu–Renault groupoid of a local homeomorphism.} Let \(f:X\to X\) be a local homeomorphism of an \(\mathbf{LCH}\) space. Define
\[
\mathcal{G}_f\coloneqq\{(x,k,y)\in X\times\mathbb{Z}\times X \mid \exists m,n\ge 0: k=n-m \ \text{and} \ f^n(x)=f^m(y)\},
\]
with \(r(x,k,y)=x\), \(s(x,k,y)=y\), \((x,k,y)^{-1}=(y,-k,x)\), and \((x,k,y)\cdot(y,\ell,z)=(x,k+\ell,z)\). Then \(\mathcal{G}_f\) is Hausdorff and \'{e}tale. It is principal if and only if no point is eventually periodic. It becomes principal on the aperiodic part.
\end{itemize}
\end{example}

Using the axioms (G1)–(G6) from Definition~\ref{def:groupoid} we first establish the invertibility of the morphisms within a groupoid.

\begin{lemma}
\((\gamma^{-1})^{-1}=\gamma\) for all \(\gamma\in\mathcal{G}\).
\end{lemma}

\begin{proof}
By (G4) applied to \(\gamma^{-1}\) we have \((\gamma^{-1})^{-1}\cdot \gamma^{-1}=s(\gamma^{-1})\) and \(\gamma^{-1}\cdot (\gamma^{-1})^{-1}=r(\gamma^{-1})\). By (G4) and (G3), \(\gamma\cdot \gamma^{-1}=r(\gamma)=s(\gamma^{-1})\) and \(\gamma^{-1}\cdot \gamma=s(\gamma)=r(\gamma^{-1})\). Thus both \((\gamma^{-1})^{-1}\) and \(\gamma\) are two-sided inverses of \(\gamma^{-1}\). In a groupoid, such an inverse is unique: if \(\alpha,\beta\in\mathcal{G}\) satisfy \(\alpha\cdot \gamma^{-1}=s(\gamma^{-1})\) and \(\gamma^{-1}\cdot \beta=r(\gamma^{-1})\), then, using (G2) and associativity (G6),
\(
\alpha
=\alpha\cdot r(\gamma^{-1})
=\alpha\cdot (\gamma^{-1}\cdot \beta)
=(\alpha\cdot \gamma^{-1})\cdot \beta
=s(\gamma^{-1})\cdot \beta
=\beta.
\)
Hence \((\gamma^{-1})^{-1}=\gamma\).
\end{proof}

Observe that \((\gamma,\gamma^{-1})\in\mathcal{G}_2\) for all \(\gamma\in\mathcal{G}\), since \(r(\gamma)=s(\gamma^{-1})\) by (G3).

\begin{lemma}
Let \(\mathcal{G}\) be a groupoid and \(\gamma\in\mathcal{G}\). Then
\begin{enumerate}[noitemsep,nolistsep]
\item for all \((\gamma,\eta)\in\mathcal{G}_2\) one has \(\gamma^{-1}\cdot(\gamma\cdot\eta)=\eta\) and \((\gamma\cdot\eta)\cdot\eta^{-1}=\gamma\).
\item \((r(\gamma),\gamma)\) and \((\gamma,s(\gamma))\) lie in \(\mathcal{G}_2\).
\item \(\gamma^{-1}\) is unique such that \((\gamma,\gamma^{-1})\in\mathcal{G}_2\), \(\gamma\cdot\gamma^{-1}=r(\gamma)\), \((\gamma^{-1},\gamma)\in\mathcal{G}_2\), and \(\gamma^{-1}\cdot\gamma=s(\gamma)\).
\end{enumerate}
\end{lemma}

\begin{proof}~
\begin{enumerate}[noitemsep,nolistsep]
\item Suppose \((\gamma,\eta)\in\mathcal{G}_2\), that is \(s(\gamma)=r(\eta)\). Then
\[
\gamma^{-1}\cdot(\gamma\cdot\eta) = (\gamma^{-1}\cdot\gamma)\cdot\eta = s(\gamma)\cdot\eta = r(\eta)\cdot \eta = \eta,
\]
where \((\gamma^{-1},\gamma)\in\mathcal{G}_2\) by (G3) and we used (G6), (G4), and (G2). Similarly,
\[
(\gamma\cdot\eta)\cdot\eta^{-1} = \gamma\cdot(\eta\cdot\eta^{-1}) = \gamma\cdot r(\eta) = \gamma,
\]
using (G6), (G4), and (G2), and \(r(\eta)=s(\gamma)\) to justify the last step.

\item Using (G4) and (G5),
\[
s\bigl(r(\gamma)\bigr) = s(\gamma\cdot\gamma^{-1}) = s(\gamma^{-1}) = r(\gamma),
\]
so \((r(\gamma),\gamma)\in\mathcal{G}_2\). Likewise,
\[
r\bigl(s(\gamma)\bigr) = r(\gamma^{-1}\cdot\gamma) = r(\gamma^{-1}) = s(\gamma),
\]
so \((\gamma,s(\gamma))\in\mathcal{G}_2\).

\item We split this proof into three parts:
\begin{itemize}[noitemsep,nolistsep]
\item \textbf{Existence:} By (G3), \((\gamma,\gamma^{-1})\) and \((\gamma^{-1},\gamma)\) lie in \(\mathcal{G}_2\), and by (G4), \(\gamma\cdot\gamma^{-1}=r(\gamma)\) and \(\gamma^{-1}\cdot\gamma=s(\gamma)\).
\item \textbf{Uniqueness from the right-unit equation:} Assume \((\gamma,\alpha)\in\mathcal{G}_2\) and \(\gamma\cdot\alpha=r(\gamma)\). Then
\[
\alpha = s(\gamma)\cdot\alpha = (\gamma^{-1}\cdot\gamma)\cdot\alpha = \gamma^{-1}\cdot(\gamma\cdot\alpha) = \gamma^{-1}\cdot r(\gamma) = \gamma^{-1},
\]
using (G2), (G4), (G6), and \(s(\gamma^{-1})=r(\gamma)\) from (G3).
\item \textbf{Uniqueness from the left-unit equation:} Assume \((\alpha,\gamma)\in\mathcal{G}_2\) and \(\alpha\cdot\gamma=s(\gamma)\). Then
\[
\alpha = \alpha\cdot r(\gamma) = \alpha\cdot(\gamma\cdot\gamma^{-1}) = (\alpha\cdot\gamma)\cdot\gamma^{-1} = s(\gamma)\cdot\gamma^{-1} = \gamma^{-1},
\]
using (G2), (G4), and (G6), and \(r(\gamma^{-1})=s(\gamma)\) from (G3).
\end{itemize}
Thus \(\gamma^{-1}\) is unique.
\end{enumerate}
\end{proof}

For later simplicial constructions, we want the domain of composition to be recorded as part of the data. Presenting a groupoid via its set of composable pairs makes the multiplication explicitly partial, allows us to form the iterated fibre products \(\mathcal{G}_n\) cleanly, and aligns with the \'{e}tale setting in which \(\mathcal{G}_2\) is the fibre product over \(\mathcal{G}_0\). This viewpoint streamlines nerves and pushforwards with topological arguments.

\begin{definition}[Groupoid {\cite[Definition~2.1.1]{Sims2018}}]
\label{def:groupoid2}
A groupoid is given by the quadruple \((\mathcal{G},\mathcal{G}_2,m,i)\) where \(\mathcal{G}\) is a set of morphisms, \(\mathcal{G}_2\subseteq \mathcal{G}\times\mathcal{G}\) is the set of composable pairs, \(m:\mathcal{G}_2\to\mathcal{G}\) multiplication, written \(m(\alpha,\beta)=\alpha\cdot\beta\), and \(i:\mathcal{G}\to\mathcal{G}\) inversion, subject to:
\begin{enumerate}[noitemsep,nolistsep]
\item[(G1')] \((\gamma^{-1})^{-1}=\gamma\) for all \(\gamma\in\mathcal{G}\).
\item[(G2')] If \((\alpha,\beta)\) and \((\beta,\gamma)\) lie in \(\mathcal{G}_2\), then \((\alpha\cdot\beta,\gamma)\) and \((\alpha,\beta\cdot\gamma)\) lie in \(\mathcal{G}_2\) and \((\alpha\cdot\beta)\cdot\gamma=\alpha\cdot(\beta\cdot\gamma)\).
\item[(G3')] For all \(\gamma\in\mathcal{G}\), \((\gamma,\gamma^{-1})\in\mathcal{G}_2\).
\item[(G4')] For all \((\gamma,\eta)\in\mathcal{G}_2\) one has \(\gamma^{-1}\cdot(\gamma\cdot\eta)=\eta\) and \((\gamma\cdot\eta)\cdot\eta^{-1}=\gamma\).
\end{enumerate}

Define the unit space by
\[
\mathcal{G}_0\coloneqq\{\gamma\cdot\gamma^{-1}\mid \gamma\in\mathcal{G}\}
=\{\gamma^{-1}\cdot\gamma\mid \gamma\in\mathcal{G}\},
\]
and the range and source maps by \(r(\gamma)\coloneqq \gamma\cdot\gamma^{-1}\) and \(s(\gamma)\coloneqq \gamma^{-1}\cdot\gamma\).
Henceforth we identify the domain of multiplication with source--range matching:
\[
(\alpha,\beta)\in\mathcal{G}_2\ \Leftrightarrow\ s(\alpha)=r(\beta).
\]
We use left--to--right composition \(\alpha\cdot\beta\).
\end{definition}

Units are derived rather than primitive in this version. The four axioms encode inversion, associativity with domain control, and the two unit laws via cancellation identities, from which \(\mathcal{G}_0\) together with \(r\) and \(s\) are defined. This presentation is economical for proofs and aligns with categorical practice, morphisms first, objects reconstructed, yet it is equivalent to the expanded axiom list used earlier. The relationship with the previous definition is as follows:

\begin{proposition}[Equivalence of presentations]
Let \(\mathcal{G}\) be a groupoid \((\mathcal{G},\mathcal{G}_0,r,s,m,i)\) and axioms (G1)–(G6) as in Definition~\ref{def:groupoid}. Then one obtains the presentation from Definition~\ref{def:groupoid2} by setting \(\mathcal{G}_2\coloneqq\{(\alpha,\beta)\in\mathcal{G}\times\mathcal{G}\mid s(\alpha)=r(\beta)\}\) and \(m(\alpha,\beta)\coloneqq \alpha\cdot\beta\), with inversion as before. Conversely, let \((\mathcal{G},\mathcal{G}_2,m,i)\) satisfy (G1')–(G4'). Define \(\mathcal{G}_0\coloneqq \{\gamma\cdot\gamma^{-1}\mid \gamma\in\mathcal{G}\}=\{\gamma^{-1}\cdot\gamma\mid \gamma\in\mathcal{G}\}\), \(r(\gamma)\coloneqq \gamma\cdot\gamma^{-1}\), and \(s(\gamma)\coloneqq \gamma^{-1}\cdot\gamma\). Then the axioms (G1)–(G6) hold with left--to--right composition, and the domain of multiplication agrees with source--range matching, that is \((\alpha,\beta)\in\mathcal{G}_2 \Leftrightarrow s(\alpha)=r(\beta)\).
\end{proposition}

\begin{proof}~
\begin{itemize}
\item \textbf{Assume we are given \((\mathcal{G},\mathcal{G}_0,r,s,m,i)\) satisfying (G1)–(G6).} Define \(\mathcal{G}_2\coloneqq\{(\alpha,\beta)\in\mathcal{G}\times\mathcal{G}\mid s(\alpha)=r(\beta)\}\) and \(m(\alpha,\beta)\coloneqq \alpha\cdot\beta\). Then:
\begin{itemize}[noitemsep,nolistsep]
\item (G1') holds by the involutivity lemma \((\gamma^{-1})^{-1}=\gamma\), which follows from (G2)–(G4) and associativity (G6).
\item (G2') holds by (G5) and (G6): if \((\alpha,\beta)\) and \((\beta,\gamma)\) lie in \(\mathcal{G}_2\), then \(s(\alpha)=r(\beta)\) and \(s(\beta)=r(\gamma)\). By (G5), \(s(\alpha\cdot\beta)=s(\beta)=r(\gamma)\) and \(r(\beta\cdot\gamma)=r(\beta)=s(\alpha)\), hence \((\alpha\cdot\beta,\gamma)\) and \((\alpha,\beta\cdot\gamma)\) lie in \(\mathcal{G}_2\), and (G6) yields \((\alpha\cdot\beta)\cdot\gamma=\alpha\cdot(\beta\cdot\gamma)\).
\item (G3') follows from (G3), which gives \((\gamma,\gamma^{-1})\in\mathcal{G}_2\) for all \(\gamma\).
\item (G4') is exactly the cancellation lemma above: for all \((\gamma,\eta)\in\mathcal{G}_2\) one has \(\gamma^{-1}\cdot(\gamma\cdot\eta)=\eta\) and \((\gamma\cdot\eta)\cdot\eta^{-1}=\gamma\).
\end{itemize}

\item \textbf{Assume we are given \((\mathcal{G},\mathcal{G}_2,m,i)\) satisfying (G1')–(G4').} Define \(\mathcal{G}_0\coloneqq \{\gamma\cdot\gamma^{-1}\mid \gamma\in\mathcal{G}\}=\{\gamma^{-1}\cdot\gamma\mid \gamma\in\mathcal{G}\}\), \(r(\gamma)\coloneqq \gamma\cdot\gamma^{-1}\), and \(s(\gamma)\coloneqq \gamma^{-1}\cdot\gamma\). We verify (G1)–(G6) and the identification of the domain of composition.
\begin{itemize}[noitemsep,nolistsep]
\item By (G3') we have \((\gamma,\gamma^{-1})\) and \((\gamma^{-1},\gamma)\) in \(\mathcal{G}_2\). Taking \(\eta=\gamma^{-1}\) in (G4') and using (G1'), we obtain \((\gamma\cdot\gamma^{-1})\cdot(\gamma^{-1})^{-1}=\gamma\), hence \(r(\gamma)\cdot\gamma=\gamma\). Taking \(\gamma=\eta^{-1}\) in the second identity of (G4') yields \(\eta\cdot(\eta^{-1}\cdot\eta)=\eta\), hence \(\gamma\cdot s(\gamma)=\gamma\). Thus (G2) holds.
\item Let \((\alpha,\beta)\in\mathcal{G}_2\) and set \(\delta\coloneqq \beta^{-1}\cdot\alpha^{-1}\). Using (G2') and (G4'),
\[
(\alpha\cdot\beta)\cdot\delta=((\alpha\cdot\beta)\cdot\beta^{-1})\cdot\alpha^{-1}=\alpha\cdot\alpha^{-1}=r(\alpha),
\]
and similarly
\[
\delta\cdot(\alpha\cdot\beta)=\beta^{-1}\cdot(\alpha^{-1}\cdot(\alpha\cdot\beta))=\beta^{-1}\cdot\beta=s(\beta).
\]
Thus \(\delta\) is a two-sided inverse of \(\alpha\cdot\beta\), hence \((\alpha\cdot\beta)^{-1}=\beta^{-1}\cdot\alpha^{-1}\).
\item For \(x\in\mathcal{G}_0\) there is \(\gamma\) with \(x=\gamma\cdot\gamma^{-1}\). Then \(x^{-1}=(\gamma\cdot\gamma^{-1})^{-1}=(\gamma^{-1})^{-1}\cdot\gamma^{-1}=\gamma\cdot\gamma^{-1}=x\), so \(x\) is fixed by inversion. Consequently \(r(x)=x\cdot x^{-1}=x\) and \(s(x)=x^{-1}\cdot x=x\), which is (G1). Also \(r(\gamma^{-1})=\gamma^{-1}\cdot(\gamma^{-1})^{-1}=\gamma^{-1}\cdot\gamma=s(\gamma)\) and \(s(\gamma^{-1})=(\gamma^{-1})^{-1}\cdot\gamma^{-1}=\gamma\cdot\gamma^{-1}=r(\gamma)\), which is (G3). Finally, (G4) holds by definition of \(r\) and \(s\).
\item Let \((\alpha,\beta)\in\mathcal{G}_2\). Then
\[
r(\alpha\cdot\beta)=(\alpha\cdot\beta)\cdot(\alpha\cdot\beta)^{-1}=(\alpha\cdot\beta)\cdot(\beta^{-1}\cdot\alpha^{-1})=\alpha\cdot\alpha^{-1}=r(\alpha),
\]
and
\[
s(\alpha\cdot\beta)=(\alpha\cdot\beta)^{-1}\cdot(\alpha\cdot\beta)=(\beta^{-1}\cdot\alpha^{-1})\cdot(\alpha\cdot\beta)=\beta^{-1}\cdot\beta=s(\beta),
\]
which is (G5).
\item Associativity on composable triples is exactly (G2'), giving (G6).
\end{itemize}

\item \textbf{\((\alpha,\beta)\in\mathcal{G}_2 \Leftrightarrow s(\alpha)=r(\beta)\).} If \((\alpha,\beta)\in\mathcal{G}_2\), then \((\alpha\cdot\beta,\beta^{-1})\in\mathcal{G}_2\) by (G2') and (G3'), hence by (G5) and (G3),
\[
s(\alpha)=s\bigl((\alpha\cdot\beta)\cdot\beta^{-1}\bigr)=s(\beta^{-1})=r(\beta).
\]
Conversely, assume \(s(\alpha)=r(\beta)\). By (G3') and (G2') applied to \((\alpha,\alpha^{-1})\) and \((\alpha^{-1},\alpha)\), we have \((\alpha,s(\alpha))\in\mathcal{G}_2\). Similarly, \((r(\beta),\beta)\in\mathcal{G}_2\). Since \(s(\alpha)=r(\beta)\), it follows that \((s(\alpha),\beta)\in\mathcal{G}_2\), and another application of (G2') to \((\alpha,s(\alpha))\) and \((s(\alpha),\beta)\) yields \((\alpha,\beta)\in\mathcal{G}_2\).
\end{itemize}
\end{proof}

\section{Isotropy}
In many applications groupoids serve as replacement without singularities for quotients by group actions, and the isotropy encodes precisely where this replacement still carries nontrivial stabiliser symmetry. Understanding the isotropy subgroupoid and the notion of principality is therefore essential for detecting when a groupoid behaves like a genuine equivalence relation, and for relating its analytical and homological invariants to those of classical free actions.

\begin{setting}
For \(x,y\in\mathcal{G}_0\) set \(\mathcal{G}(x,y)\coloneqq\{\gamma\in\mathcal{G}\mid r(\gamma)=x,\ s(\gamma)=y\}=r^{-1}(x)\cap s^{-1}(y)\). For \(U,V\subseteq\mathcal{G}\) write \(UV\coloneqq\{\alpha\cdot\beta\mid \alpha\in U,\ \beta\in V,\ s(\alpha)=r(\beta)\}\).
\end{setting}

\begin{definition}[Isotropy subgroupoid {\cite[Chapter~2.2]{Sims2018}}]
\(\operatorname{Iso}(\mathcal{G})\coloneqq\bigsqcup_{x\in\mathcal{G}_0}\mathcal{G}(x,x)=\{\gamma\in\mathcal{G}\mid r(\gamma)=s(\gamma)\}\).
It is a subgroupoid, a group bundle over \(\mathcal{G}_0\) with fibres \(\mathcal{G}_x\coloneqq\mathcal{G}(x,x)\). In particular \(\mathcal{G}_0\subseteq \operatorname{Iso}(\mathcal{G})\).
\end{definition}

\begin{lemma}[{\cite[Lemma~2.2.1]{Sims2018}}]\label{lem:principal-iff-iso-equals-units}
\(\mathcal{G}\) is principal if and only if \(\operatorname{Iso}(\mathcal{G})=\mathcal{G}_0\).
\end{lemma}

\begin{proof}
If \(\mathcal{G}\) is principal, then for every \(x\in\mathcal{G}_0\) the isotropy group \(\mathcal{G}_x\) equals \(\{x\}\). Let \(\gamma\in\operatorname{Iso}(\mathcal{G})\). Then \(r(\gamma)=s(\gamma)=x\) for some \(x\in\mathcal{G}_0\), hence \(\gamma\in\mathcal{G}_x=\{x\}\), so \(\gamma=x\in\mathcal{G}_0\). Thus \(\operatorname{Iso}(\mathcal{G})\subseteq\mathcal{G}_0\). The reverse inclusion holds because for each \(x\in\mathcal{G}_0\) we have \(r(x)=x=s(x)\) by (G1). Hence \(\operatorname{Iso}(\mathcal{G})=\mathcal{G}_0\). Conversely, suppose \(\operatorname{Iso}(\mathcal{G})=\mathcal{G}_0\). Fix \(x\in\mathcal{G}_0\) and \(\gamma\in\mathcal{G}\) with \(r(\gamma)=s(\gamma)=x\). Then \(\gamma\in\operatorname{Iso}(\mathcal{G})=\mathcal{G}_0\). By (G1), \(r(\gamma)=\gamma\), and since also \(r(\gamma)=x\), it follows that \(\gamma=x\). Therefore \(\mathcal{G}_x=\{x\}\) for all \(x \in \mathcal{G}_0\), and \(\mathcal{G}\) is principal.
\end{proof}

\begin{example}[Isotropy in a transformation groupoid {\cite[Example~2.2.2]{Sims2018}}]
Let a discrete group \(\Gamma\) act on a set or on an \(\mathbf{LCH}\) space \(X\) by homeomorphisms. In \(\Gamma\ltimes X\) one has
\[
\operatorname{Iso}(\Gamma\ltimes X)=\bigsqcup_{x\in X}\{(g,x)\mid g\cdot x=x\}\ \cong\ \bigsqcup_{x\in X}\Gamma_x,
\]
where \(\Gamma_x\coloneqq\{g\in\Gamma \mid g\cdot x=x\}\) are the stabilisers. Hence \(\Gamma\ltimes X\) is principal if and only if the action is free.
\end{example}

\begin{lemma}[{\cite[Lemma~2.2.3]{Sims2018}}]\label{lem:Ad-conjugacy-iso}
For \(\gamma\in\mathcal{G}\) the map \(\mathrm{Ad}_\gamma:\mathcal{G}_{s(\gamma)}\to\mathcal{G}_{r(\gamma)}\), \(\alpha\mapsto \gamma\cdot\alpha\cdot\gamma^{-1}\) is a group isomorphism.
\end{lemma}

\begin{proof}~
\begin{itemize}
\item \textbf{Well definedness:} Let \(\alpha\in\mathcal{G}_{s(\gamma)}\). Then \(r(\alpha)=s(\gamma)\) and \(s(\alpha)=s(\gamma)\). By (G5), \(r(\gamma\cdot\alpha)=r(\gamma)\) and \(s(\gamma\cdot\alpha)=s(\alpha)=s(\gamma)\). Since \(r(\gamma^{-1})=s(\gamma)\) by (G3), we have \((\gamma\cdot\alpha,\gamma^{-1})\in\mathcal{G}_2\), and another application of (G5) gives \(r((\gamma\cdot\alpha)\cdot\gamma^{-1})=r(\gamma)\) and \(s((\gamma\cdot\alpha)\cdot\gamma^{-1})=s(\gamma^{-1})=r(\gamma)\), so \(\mathrm{Ad}_\gamma(\alpha)\in\mathcal{G}_{r(\gamma)}\).
\item \textbf{Homomorphism property:} For \(\alpha,\beta\in\mathcal{G}_{s(\gamma)}\) the pairs \((\gamma,\alpha)\), \((\alpha,\beta)\), and \((\beta,\gamma^{-1})\) are composable. By associativity (G6),
\[
\mathrm{Ad}_\gamma(\alpha)\cdot \mathrm{Ad}_\gamma(\beta)
=(\gamma\cdot\alpha\cdot\gamma^{-1})\cdot(\gamma\cdot\beta\cdot\gamma^{-1})
=\gamma\cdot\alpha\cdot(\gamma^{-1}\cdot\gamma)\cdot\beta\cdot\gamma^{-1}
=\gamma\cdot(\alpha\cdot\beta)\cdot\gamma^{-1}
=\mathrm{Ad}_\gamma(\alpha\cdot\beta),
\]
using (G4) in the third equality.
\item \textbf{Bijectivity:} Consider \(\mathrm{Ad}_{\gamma^{-1}}:\mathcal{G}_{r(\gamma)}\to\mathcal{G}_{s(\gamma)}\). For \(\alpha\in\mathcal{G}_{s(\gamma)}\),
\[
\mathrm{Ad}_{\gamma^{-1}}(\mathrm{Ad}_\gamma(\alpha))
=\gamma^{-1}\cdot(\gamma\cdot\alpha\cdot\gamma^{-1})\cdot\gamma
=(\gamma^{-1}\cdot\gamma)\cdot\alpha\cdot(\gamma^{-1}\cdot\gamma)
=s(\gamma)\cdot\alpha\cdot s(\gamma)
=\alpha,
\]
using (G6), (G4), and (G2). Similarly, for \(\beta\in\mathcal{G}_{r(\gamma)}\),
\[
\mathrm{Ad}_\gamma(\mathrm{Ad}_{\gamma^{-1}}(\beta))
=\gamma\cdot(\gamma^{-1}\cdot\beta\cdot\gamma)\cdot\gamma^{-1}
=(\gamma\cdot\gamma^{-1})\cdot\beta\cdot(\gamma\cdot\gamma^{-1})
=r(\gamma)\cdot\beta\cdot r(\gamma)
=\beta,
\]
using (G6), (G4), and (G2). Thus \(\mathrm{Ad}_{\gamma^{-1}}\) is the inverse of \(\mathrm{Ad}_\gamma\).
\end{itemize}
Therefore \(\mathrm{Ad}_\gamma\) is a group isomorphism.
\end{proof}

\section{Topological Groupoids}
Topological structure allows groupoids to encode convergence, compactness, and local symmetries, unifying groups, group actions, and equivalence relations within a single analytic–geometric framework. It also captures pathological quotients faithfully -- for instance the orbit spaces in Example~\ref{ex:orbits} -- while retaining continuous structure maps that are crucial for later homological constructions.

\begin{definition}[Topological groupoids {\cite[§2.1]{Komura2020}}]\label{def:topological-groupoid}
A topological groupoid is a groupoid \((\mathcal{G},\mathcal{G}_0,r,s,m,i,u)\) equipped with topologies on \(\mathcal{G}\) and \(\mathcal{G}_0\) such that:
\begin{enumerate}[noitemsep,nolistsep]
  \item[(T1)] \(\mathcal{G}_0\subseteq \mathcal{G}\) is the unit space, endowed with the subspace topology.
  \item[(T2)] \(r,s:\mathcal{G}\to\mathcal{G}_0\) are continuous.
  \item[(T3)] \(\mathcal{G}_2\coloneqq\{(\gamma,\eta)\in\mathcal{G}\times\mathcal{G}\mid s(\gamma)=r(\eta)\}\) carries the subspace topology from \(\mathcal{G}\times\mathcal{G}\), and multiplication \(m:\mathcal{G}_2\to\mathcal{G}\), \((\gamma,\eta)\mapsto \gamma\cdot\eta\), is continuous.
  \item[(T4)] The inversion map \(i:\mathcal{G}\to\mathcal{G}\), \(i(\gamma)\coloneqq \gamma^{-1}\), is continuous.
  \item[(T5)] The unit map \(u:\mathcal{G}_0\to\mathcal{G}\), \(u(x)\coloneqq 1_x\), is continuous.
\end{enumerate}
\end{definition}


Subgroupoids isolate controlled pieces of a groupoid -- reductions to invariant subspaces, isotropy, kernels/images of homomorphisms. They are building blocks for exact constructions.

\begin{definition}[Subgroupoid {\cite[§2.1]{Komura2020}}]
Let \(\mathcal{G}\) be a groupoid. A subgroupoid consists of subsets \(\mathcal{H}\subseteq\mathcal{G}\) and \(\mathcal{H}_0\subseteq\mathcal{G}_0\) together with the restricted structure maps
\(
r\vert^{\mathcal{H}_0}_{\mathcal{H}}:\mathcal{H}\to\mathcal{H}_0,
s\vert^{\mathcal{H}_0}_{\mathcal{H}}:\mathcal{H}\to\mathcal{H}_0,
m\vert^{\mathcal{H}}_{\mathcal{H}_2}:\mathcal{H}_2\to\mathcal{H},
(i)|^{\mathcal{H}}_{\mathcal{H}}:\mathcal{H}\to\mathcal{H},
\)
where \(\mathcal{H}_2\coloneqq\{(\alpha,\beta)\in\mathcal{H}\times\mathcal{H}\mid s(\alpha)=r(\beta) \}\), such that:
\begin{enumerate}[noitemsep,nolistsep]
  \item[(S1)] \(\mathcal{H}_0=\mathcal{H}\cap\mathcal{G}_0\) and \(r(\mathcal{H})\cup s(\mathcal{H})\subseteq\mathcal{H}_0\).
  \item[(S2)] If \((\alpha,\beta)\in\mathcal{H}_2\), then \(\alpha\cdot\beta\in\mathcal{H}\).
  \item[(S3)] If \(\gamma\in\mathcal{H}\), then \(\gamma^{-1}\in\mathcal{H}\).
  \item[(S4)] For all \(x\in\mathcal{H}_0\) one has \(r(x)=x=s(x)\) and \(x\in\mathcal{H}\).
\end{enumerate}
\end{definition}

\begin{remark}
    We suppress the notation of the restriction and corestriction, due to readability.
\end{remark}

\begin{remark}
The subgroupoid \(\mathcal{H}\) is wide if \(\mathcal{H}_0=\mathcal{G}_0\). It is normal if it is wide and for every \(\gamma\in\mathcal{G}\) with \(s(\gamma)=x\), \(r(\gamma)=y\) one has \(\gamma \mathcal{H}_x \gamma^{-1}\subseteq \mathcal{H}_y,\) where \(\mathcal{H}_x\coloneqq\{\eta\in\mathcal{H}\mid r(\eta)=s(\eta)=x\}.\) Since the same inclusion holds with \(\gamma\) replaced by \(\gamma^{-1}\), normality is equivalent to \(\gamma \mathcal{H}_x \gamma^{-1}= \mathcal{H}_y\) for all such \(\gamma\).
\end{remark}

\begin{proposition}[Quotient groupoid {\cite[§11.3.1]{Brown2006}}]
Let \(\mathcal{N}\subseteq\mathcal{G}\) be a wide normal subgroupoid whose morphisms are isotropy elements, that is \(\{\eta\in\mathcal{N}\mid r(\eta)=x,\ s(\eta)=y\}=\varnothing\) for \(x\neq y\). The quotient groupoid \(\mathcal{G}/\mathcal{N}\) is defined by:
\begin{enumerate}[noitemsep,nolistsep]
  \item[(Q1)] \textbf{Objects:} \((\mathcal{G}/\mathcal{N})_0=\mathcal{G}_0\).
  \item[(Q2)] \textbf{Morphisms:} for \(\gamma\in\mathcal{G}\), write \(\mathcal{N}_{r(\gamma)}\gamma\mathcal{N}_{s(\gamma)}\coloneqq\{n\gamma n' \mid n\in\mathcal{N}_{r(\gamma)},\ n'\in\mathcal{N}_{s(\gamma)}\}\) and denote this double coset by \([\gamma]\). Then \((\mathcal{G}/\mathcal{N})(x,y)\coloneqq\{[\gamma]\mid \gamma\in\mathcal{G}(x,y)\}\).
  \item[(Q3)] \textbf{Structure maps:} \(r([\gamma])\coloneqq r(\gamma)\), \(s([\gamma])\coloneqq s(\gamma)\), \([g]^{-1}\coloneqq[g^{-1}]\), and if \(s(g)=r(h)\) then \([\gamma]\cdot[\eta]\coloneqq[g\cdot h]\).
\end{enumerate}
\end{proposition}

\begin{proof}
Define \(g\sim g'\) if \(g'\in \mathcal{N}_{r(g)}g \mathcal{N}_{s(g)}\).
Then \(\sim\) is an equivalence relation on \(\mathcal{G}\). Reflexivity follows from \(g=1_{r(g)} g 1_{s(g)}\). If \(g'=a g b\) with \(a\in\mathcal{N}_{r(g)}\) and \(b\in\mathcal{N}_{s(g)}\), then \(g = a^{-1} g' b^{-1}\), with \(a^{-1}\in\mathcal{N}_{r(g)}\) and \(b^{-1}\in\mathcal{N}_{s(g)}\); hence the relation \(g'\in \mathcal{N}_{r(g)} g \mathcal{N}_{s(g)}\) is symmetric. If \(g'=agb\) and \(g''=cg'd\) with \(a\in\mathcal{N}_{r(g)}\), \(b\in\mathcal{N}_{s(g)}\), \(c\in\mathcal{N}_{r(g')}\), \(d\in\mathcal{N}_{s(g')}\), then \(r(g')=r(g)\) and \(s(g')=s(g)\) because \(a,b\) are isotropy at \(r(g)\), \(s(g)\). Hence \(c\in\mathcal{N}_{r(g)}\) and \(d\in\mathcal{N}_{s(g)}\). Since each \(\mathcal{N}_x\) is a subgroup of \(\mathcal{G}_x\), we have \(c a\in\mathcal{N}_{r(g)}\) and \(b d\in\mathcal{N}_{s(g)}\), so \(g''=c g' d=c(agb) d=(ca) g (bd)\in \mathcal{N}_{r(g)} g \mathcal{N}_{s(g)}\). Thus \(g\sim g''\) and the transitivity follows.

\begin{itemize}[noitemsep,nolistsep]
    \item \textbf{Source, range, and inversion are well defined.} If \(g'=a g b\) with \(a\in\mathcal{N}_{r(g)}\), \(b\in\mathcal{N}_{s(g)}\), then \(r(g')=r(g)\) and \(s(g')=s(g)\). Thus \(r([g])\) and \(s([g])\) are well defined.

    Moreover \(g'^{-1}=b^{-1} g^{-1} a^{-1}\) with \(b^{-1}\in\mathcal{N}_{s(g)}\), \(a^{-1}\in\mathcal{N}_{r(g)}\), hence \([g'^{-1}]=[g^{-1}]\).

    \item \textbf{Composition is well defined.} Assume \(s(g)=r(h)\), and choose representatives \(g_1=a g b\), \(h_1=c h d\) with \(a\in\mathcal{N}_{r(g)}\), \(b\in\mathcal{N}_{s(g)}=\mathcal{N}_{r(h)}\), \(c\in\mathcal{N}_{r(h)}\), \(d\in\mathcal{N}_{s(h)}\). Then \(g_1\cdot h_1\) is defined since \(s(g_1)=s(g)=r(h)=r(h_1)\). Set \(n\coloneqq b c\in \mathcal{N}_{s(g)}\). Then
    \(
    g_1\cdot h_1
    = a g (b c) h d
    = a g n h d.
    \)
    By normality of \(\mathcal{N}\), \(g n g^{-1}\in\mathcal{N}_{r(g)}\). Set \(a'\coloneqq a(g n g^{-1})\in\mathcal{N}_{r(g)}\). Then
    \(
    a'(g\cdot h)d
    = a(g n g^{-1})(g h)d
    = a g n h d
    = g_1\cdot h_1,
    \)
    so \([g_1\cdot h_1]=[g\cdot h]\).

    \item \textbf{Units and associativity.} For \(x\in\mathcal{G}_0\), \([1_x]\) is a two-sided unit \([1_{r(g)}]\cdot[g]=[1_{r(g)}\cdot g]=[g]\) and \([g]\cdot[1_{s(g)}]=[g\cdot 1_{s(g)}]=[g]\). Associativity follows from associativity in \(\mathcal{G}\) \(([g]\cdot[h])\cdot[k]=[(g\cdot h)\cdot k]=[g\cdot (h\cdot k)]=[g]\cdot([h]\cdot[k])\).
\end{itemize}
\end{proof}

\begin{remark}
If $\mathcal{G}$ is a topological groupoid and $\mathcal{N}$ is a closed normal wide subgroupoid, endow $\mathcal{G}/\mathcal{N}$ with the quotient topology making the functor $\mathcal{G}\to \mathcal{G}/\mathcal{N}$ continuous.
\end{remark}

\begin{setting}
For units $x,y\in\mathcal{G}_0$ we set
$\mathcal{G}(x,y)\coloneqq\{\gamma\in\mathcal{G}\mid r(\gamma)=x,\ s(\gamma)=y\}$,
the set of morphisms from $y$ to $x$. Composition restricts to $\mathcal{G}(x,y)\times\mathcal{G}(y,z)\to\mathcal{G}(x,z)$, $(\alpha,\beta)\mapsto\alpha\cdot\beta$. $\mathcal{G}(x,y)$ and $\mathcal{N}_x$ carry the subspace topology from $\mathcal{G}$.
\end{setting}

The following examples form the basic operations for constructing and analysing groupoids. The pair groupoid \(X\times X\) encodes the space itself. The transformation groupoid \(\Gamma\ltimes X\) models orbit spaces of actions while remaining \'{e}tale for discrete \(\Gamma\). Reductions \(\mathcal{G}\vert_U\) implement localisation along units, and quotients by wide normal isotropy subgroupoids collapse internal symmetries. Such quotients are needed for functoriality, Morita invariance, and homology.

\begin{example}~
\label{ex:groupoids}
\begin{itemize}[noitemsep,nolistsep]
  \item \textbf{Topological groupoid.} Let \(X\) be a topological space. Set \(\mathcal{G}=X\times X\), \(\mathcal{G}_0=X\), \(r(x,y)=x\), \(s(x,y)=y\), inversion \((x,y)^{-1}=(y,x)\), and composition \((x,y)\cdot(y,z)=(x,z)\). All structure maps are continuous, so \(\mathcal{G}\) is a topological groupoid. It is \'{e}tale iff \(X\) is discrete. Otherwise the projections are not local homeomorphisms. If $X$ is discrete, then for each $(x,y)\in\G$ the singleton $\{(x,y)\}$ is open in $X\times X$, and
  \(r\big|_{\{(x,y)\}} : \{(x,y)\}\to\{x\},\)
  \(s\big|_{\{(x,y)\}} : \{(x,y)\}\to\{y\},\)
  are homeomorphisms onto open subsets of $X$. Hence $r$ and $s$ are local homeomorphisms, so $\G$ is \'{e}tale. Conversely, suppose $\G$ is \'{e}tale. Then in particular $r$ is a local homeomorphism. Fix $x\in X$ and consider the arrow $(x,x)\in\G$. By local homeomorphy of $r$ there exist open neighbourhoods $U\subseteq X\times X$ of $(x,x)$ and $V\subseteq X$ of $x$ such that \(r\big|_U : U\to V\) is a homeomorphism. Since $r$ is the projection onto the first coordinate, the restriction $r\big|_U$ is bijective, so $U$ is the graph of a continuous map $f:V\to X$: \(U = \{(v,f(v)) \mid v\in V\}.\) Because $U$ is open in $X\times X$ and $(x,x)\in U$, there exist open neighbourhoods $W_1,W_2\subseteq X$ of $x$ such that $W_1\times W_2\subseteq U$. Let $w\in W_2$. Then for every $v\in W_1$ we have $(v,w)\in W_1\times W_2\subseteq U$, so $(v,w)=(v,f(v))$ and hence $w=f(v)$. Thus $w$ does not depend on $v$, and in particular for $v=x$ we get $w=f(x)$. Since also $(x,x)\in U$, we have $f(x)=x$, so $w=x$. Therefore every $w\in W_2$ equals $x$, and hence $W_2=\{x\}$ is open in $X$. As $x\in X$ was arbitrary, every singleton $\{x\}$ is open, so $X$ is discrete.

  \item \textbf{\'{E}tale groupoid.} Let a discrete group \(\Gamma\) act by homeomorphisms on a locally compact Hausdorff space \(X\). Set \(\mathcal{G}=\Gamma\ltimes X\), \(\mathcal{G}_0=X\), \(r(g,x)=g\cdot x\), \(s(g,x)=x\), inversion \((g,x)^{-1}=(g^{-1},g\cdot x)\), and composition \((h,g\cdot x)\cdot(g,x)=(hg,x)\). Since \(\Gamma\) is discrete, \(r\) and \(s\) are local homeomorphisms. Hence \(\Gamma\ltimes X\) is \'{e}tale.

  \item \textbf{Subgroupoid.} Let \(\mathcal{G}\) be a topological groupoid and \(U\subseteq \mathcal{G}_0\). Define the reduction
  \[
  \mathcal{G}\vert_U\coloneqq\{\gamma\in\mathcal{G}\mid r(\gamma)\in U, s(\gamma)\in U\}
  \]
  with unit space \(U\) and the induced structure maps \(r\vert_{\mathcal{G}\vert_U}\), \(s\vert_{\mathcal{G}\vert_U}\), \(\cdot\vert_{(\mathcal{G}\vert_U)_2}\), and \(^{-1}\vert_{\mathcal{G}\vert_U}\). Then \(\mathcal{G}\vert_U\) is a subgroupoid of \(\mathcal{G}\). If \(U\) is open or closed, it is an open or closed topological subgroupoid.

  \item \textbf{Quotient groupoid.} View a topological group \(G\) as a one-object groupoid with \((\mathcal{G},\mathcal{G}_0)=(G,\{e\})\). Let \(N\trianglelefteq G\) be a closed normal subgroup. Then \(N\subseteq G\) is a wide normal subgroupoid whose morphisms are isotropy elements, and the quotient groupoid \(\mathcal{G}/N\) is the one-object groupoid \((G/N,\{[e]\})\) with the usual quotient topology and structure maps. Here \([g]\cdot[h]=[gh]\), \([g]^{-1}=[g^{-1}]\), and \(r([g])=s([g])=[e]\).
\end{itemize}
\end{example}

\begin{remark}
For a left group action \(\Gamma \rhd X\) by homeomorphisms, \(\Gamma\ltimes X\) is the transformation groupoid. The object is \(X\) and the morphisms are \(\Gamma\times X\), with structure maps as in Example~\ref{ex:groupoids}.
\end{remark}

One might expect the unit space to be closed in a topological groupoid. In general this fails. However, it is closed if the arrow space is Hausdorff, but the converse needs not to hold.
The converse implication fails for general groupoids in \(\mathbf{Top}\) if one does not assume that
\(\mathcal{G}_0\) is Hausdorff: the unit groupoid on a non-Hausdorff space \(X\) satisfies
\(\mathcal{G}=\mathcal{G}_0=X\), so \(\mathcal{G}_0\) is closed in \(\mathcal{G}\) but \(\mathcal{G}\) is not Hausdorff. This example is excluded by the convention in \cite{Sims2018}, where a topological groupoid
is required to have Hausdorff unit space.

\begin{proposition}
If \(\mathcal{G}\) is Hausdorff, then \(\mathcal{G}_0\) is closed in \(\mathcal{G}\).
\end{proposition}

\begin{proof}
Consider the continuous map \(H:\mathcal{G}\to\mathcal{G}\times\mathcal{G}\) given by
\(H(\gamma)\coloneqq(\gamma,r(\gamma))\).
Let \(\Delta_{\mathcal{G}}\coloneqq\{(\eta,\eta)\mid \eta\in\mathcal{G}\}\) be the diagonal.
Since \(\mathcal{G}\) is Hausdorff, \(\Delta_{\mathcal{G}}\) is closed in \(\mathcal{G}\times\mathcal{G}\).
An arrow \(\gamma\) is a unit if and only if \(\gamma=r(\gamma)\), hence
\(\mathcal{G}_0=H^{-1}(\Delta_{\mathcal{G}})\).
Therefore \(\mathcal{G}_0\) is closed in \(\mathcal{G}\).
\end{proof}

\begin{proposition}\label{lem:G0-closed-iff-Hausdorff}
Let \(\mathcal G\) be a topological groupoid such that \(\mathcal G_0\subseteq \mathcal G\) is Hausdorff in the relative topology. Then \(\mathcal G\) is Hausdorff if and only if \(\mathcal G_0\) is closed in \(\mathcal G\).
\end{proposition}

\begin{proof}~
\begin{itemize}[noitemsep,nolistsep]
\item \textbf{Assume that \(\mathcal G\) is Hausdorff.} Let \((x_i)_{i\in I}\) be a net in \(\mathcal G_0\) such that \(x_i\to \gamma\in\mathcal G\). Since \(r\) is continuous and \(r(x_i)=x_i\) for all \(i\), we have \(x_i=r(x_i)\to r(\gamma)\in\mathcal G_0\). By uniqueness of limits in a Hausdorff space, \(\gamma=r(\gamma)\), hence \(\gamma\in\mathcal G_0\). Thus \(\mathcal G_0\) is closed.

\item \textbf{Assume that \(\mathcal G_0\) is closed in \(\mathcal G\).} To show that \(\mathcal G\) is Hausdorff, it suffices to show that every convergent net has a unique limit.
Let \((\gamma_i)_{i\in I}\) be a net in \(\mathcal G\) with \(\gamma_i\to \alpha\) and \(\gamma_i\to \beta\). First, continuity of \(r:\mathcal G\to \mathcal G_0\) gives
\(r(\gamma_i)\to r(\alpha)\) and \(r(\gamma_i)\to r(\beta)\) in the Hausdorff space \(\mathcal G_0\), so \(r(\alpha)=r(\beta)\). Hence \((\alpha^{-1},\beta)\in\mathcal G_2\). Next, since inversion is continuous, \(\gamma_i^{-1}\to \alpha^{-1}\). Therefore \((\gamma_i^{-1},\gamma_i)\to (\alpha^{-1},\beta)\) in \(\mathcal G\times\mathcal G\), and hence in the subspace \(\mathcal G_2\). By continuity of multiplication, \(\gamma_i^{-1}\gamma_i \to \alpha^{-1}\beta\) in \(\mathcal G\). But \(\gamma_i^{-1}\gamma_i = s(\gamma_i)\in\mathcal G_0\) for all \(i\). Since \(\mathcal G_0\) is closed in \(\mathcal G\), the limit \(\alpha^{-1}\beta\) belongs to \(\mathcal G_0\). Thus \(\alpha^{-1}\beta\) is a unit, and hence \(\beta = \alpha(\alpha^{-1}\beta)=\alpha\). So limits are unique, and \(\mathcal G\) is Hausdorff.
\end{itemize}
\end{proof}

\begin{remark}
In \cite[Lemma~2.3.2]{Sims2018} the statement is proved under the standing hypotheses of \cite[Definition~2.3.1]{Sims2018}, in particular assuming local compactness. However, those uses are inessential for the implication itself. Indeed, the argument only requires that the structure maps \(s,r,m,(-)^{-1},u\) are continuous and that \(\mathcal G_0\) is Hausdorff in the relative topology.
\end{remark}

\section{\'{E}tale Groupoids}
\label{chapter:etalegroupoids}
\'{E}tale groupoids occupy the middle ground between the purely topological and the smooth settings. They retain enough local regularity to support constructions such as bisections, \'{e}tale sheaves, nerves and counting measures, while remaining flexible enough to model orbit and equivalence–relation dynamics arising from local homeomorphisms and actions of discrete groups. A groupoid can be summarised by the diagram \cite[§1]{crainic1999homology}:
\[
\G_2=\G\stimesr\G
\xrightarrow{m} \G
\xrightarrow{i} \G
\overset{r}{\underset{s}{\rightrightarrows}} \G_0
\xrightarrow{u} \G,
\]
where \(s,r:\G\to\G_0\) are the source and range maps, \(u(x)\coloneqq 1_x\) inserts the identity morphism at \(x\in\G_0\), \(i(g)\coloneqq g^{-1}\) is inversion, and \(m(g,h)\coloneqq g\cdot h\) is defined exactly when \(s(g)=r(h)\), that is, \((g,h)\in\G_2\). We write \(g:x\to y\) for \(s(g)=x\) and \(r(g)=y\). A topological groupoid is obtained by equipping \(\G_0\) and \(\G\) with topologies and requiring all structure maps in the diagram above to be continuous. In the smooth (Lie) case, \(\G_0\) and \(\G\) are smooth manifolds, the structure maps are smooth, and one assumes that \(s\) and \(r\) are submersions so that the fibre product \(\G\stimesr\G\) is again a manifold \cite[§1]{crainic1999homology}. \'{E}taleness is the topological analogue of this local regularity: we require \(s\) and \(r\) to be local homeomorphisms. This forces the fibres of \(s\) and \(r\) to be discrete, aligns \'{e}tale groupoids with transformation groupoids of actions of discrete groups and equivalence–relation groupoids of local homeomorphisms, and at the same time preserves enough categorical structure for the homological constructions in Chapter~\ref{chap:moore-homology}.

\begin{definition}[\'{E}tale groupoids {\cite[Definition~1.1]{crainic1999homology}}]
\label{def:etale-groupoid}
Let $\G$ be a topological groupoid. The groupoid \(\G\) is called \'{e}tale if the source map \(s:\G\to\G_{0}\) is a local homeomorphism. In this case the range map \(r=s\circ i\) is also a local homeomorphism.
\end{definition}

\begin{remark}
    Since inversion \(i\) is a homeomorphism and \(r=s\circ i\), this is equivalent to requiring that the range map \(r:\G\to\G_{0}\) is a local homeomorphism. Many authors state the condition as: \(r\) and \(s\) are local homeomorphisms. In particular, for all \(g\in\G\) there is an open neighbourhood  \(U\subset\G\) with \(g\in U\) such that \(s|_{U}: U\xrightarrow{\cong} s(U)\) and \(r|_{U}: U\xrightarrow{\cong} r(U)\) are homeomorphisms.
\end{remark}

Such a set \(U\) is called a local bisection. In the smooth or Lie setting the same definition applies with local diffeomorphisms in place of local homeomorphisms. Direct limits assemble compatible local data around a point. Two local maps represent the same element once they agree on some smaller neighbourhood.

\begin{definition}[Direct limit]
Fix \(x\in\G_0\) and let \(\mathcal{N}(x)\) denote the directed set of open neighbourhoods of \(x\), ordered by reverse inclusion. For each \(U\in\mathcal{N}(x)\) set
\(\operatorname{LH}(U,\G_0)\coloneqq\{\varphi:U\to\G_0 \mid \varphi \text{ is a local homeomorphism}\}\). Whenever \(U'\subseteq U\) we have the restriction map
\(\rho_{U,U'}: \operatorname{LH}(U,\G_0)\to\operatorname{LH}(U',\G_0)\), \(\varphi\mapsto\varphi|_{U'}\). This yields a direct system \(\bigl(\operatorname{LH}(U,\G_0),\rho_{U,U'}\bigr)_{U\in\mathcal{N}(x)}\) in the category of sets. Its direct limit is the quotient
\[
\varinjlim_{U\in\mathcal{N}(x)}\operatorname{LH}(U,\G_0)
\cong
\Bigl(\bigsqcup_{U\in\mathcal{N}(x)}\operatorname{LH}(U,\G_0)\Bigr)\Big/\!\sim,
\]
where \((U,\varphi)\sim(U',\varphi')\) if there exists an open set \(W\subseteq U\cap U'\) with \(x\in W\) and \(\varphi|_{W}=\varphi'|_{W}\).
\end{definition}

More generally, a direct system in a category \(\mathcal{C}\) consists of a directed set \((I,\leq)\), a family of objects \((A_i)_{i\in I}\), and morphisms \(\phi_{ij}:A_i\to A_j\) for all \(i\leq j\) such that \(\phi_{ii}=\mathrm{id}_{A_i}\) and \(\phi_{jk}\circ\phi_{ij}=\phi_{ik}\) whenever \(i\leq j\leq k\). In particular, the family \(\bigl(\operatorname{LH}(U,\G_0)\bigr)_{U\in\mathcal{N}(x)}\), together with the restriction maps \(\rho_{U,U'}:\operatorname{LH}(U,\G_0)\to\operatorname{LH}(U',\G_0)\) for \(U'\subseteq U\), is a direct system in \(\mathbf{Set}\) indexed by \(\mathcal{N}(x)\). In an \'{e}tale groupoid, every arrow sufficiently close to a unit arises from a local bisection, so its local effect is recorded by the germ of a local homeomorphism of the unit space, that is, by an element of this direct limit.

\begin{definition}
Let \(\G\) be an \'{e}tale groupoid and let \(g\in\G\) with \(s(g)=x\) and \(r(g)=y\).
Choose an open bisection \(B\subseteq \G\) with \(g\in B\), and put \(U\coloneqq s(B)\subseteq \G_0\).
Let \(\sigma:U\to \G\) be the inverse of \(s|_B:B\to U\), so that
\(\sigma(x)=g\) and \(s\circ\sigma=\mathrm{id}_U\).
Set \(V\coloneqq r(B)\); then \(V\) is open in \(\G_0\) and
\(
\varphi_{(U,\sigma)}\coloneqq r\circ\sigma:U\to V
\)
is a homeomorphism (hence a local homeomorphism) sending \(x\) to \(y\).

Let \(\Sigma_g\) be the set of all such pairs \((U,\sigma)\) arising from open bisections \(B\ni g\).
Declare \((U,\sigma)\sim (U',\sigma')\) if and only if there exists an open neighbourhood
\(W\subseteq U\cap U'\) of \(x\) such that
\(\varphi_{(U,\sigma)}|_W=\varphi_{(U',\sigma')}|_W\).
The germ of \(g\) at \(x\) is the equivalence class
\[
\widetilde g \coloneqq [(U,\sigma)]
\in \varinjlim_{W\in \mathcal{N}(x)}\operatorname{LH}(W,\G_0).
\]
\end{definition}

In the smooth setting replace local homeomorphisms by local diffeomorphisms.

\begin{remark}
This construction is well defined and depends only on \(g\). Since \(s\) is a local homeomorphism at \(g\), there exist an open neighbourhood \(N\subseteq\G\) of \(g\) and an open set \(W_0\subseteq\G_0\) such that \(s|_N:N\xrightarrow{\cong} W_0\). For any \((U,\sigma),(U',\sigma')\in\Sigma_g\) we may shrink \(U\) and \(U'\) so that \(\sigma(U)\subseteq N\) and \(\sigma'(U')\subseteq N\). On \(W\coloneqq U\cap U'\cap W_0\) we then have \(\sigma|_W = (s|_N)^{-1}|_W = \sigma'|_W\), hence \((U,\sigma)\sim(U',\sigma')\) and \(\varphi_{(U,\sigma)}|_W=\varphi_{(U',\sigma')}|_W\). Thus the germ \(\widetilde g\) is independent of the choice of \((U,\sigma)\in\Sigma_g\).
\end{remark}

\begin{remark}[Basic properties {\cite[§1.2]{crainic1999homology}}]
\leavevmode
\begin{enumerate}[label=(\alph*),nosep]
\item Since \(\G\) is \'{e}tale, \(s\) is a local homeomorphism. After shrinking \(U\), there exists a local section \(\sigma:U\to\G\) of \(s\) with \(\sigma(x)=g\) and \(s\circ\sigma=\mathrm{id}_U\). The inversion map \(\G\to\G\), \(h\mapsto h^{-1}\), is a homeomorphism and \(r=s\circ(i_{\G})\), so \(r\) is a local homeomorphism as well. In particular, \(r|_{\sigma(U)}:\sigma(U)\to V\) is a homeomorphism onto the open set \(V\coloneqq r(\sigma(U))\subseteq\G_0\). Consequently the representative \(\varphi_{(U,\sigma)}=r\circ\sigma:U\to V\) of \(\widetilde g\) is a local homeomorphism.
\item For the unit \(1_x\in\G\) we may take \(U\subseteq\G_0\) and \(\sigma(u)\coloneqq 1_u\) for \(u\in U\). Then \(s\circ\sigma=\mathrm{id}_U\) and \(r\circ\sigma=\mathrm{id}_U\), so the germ \(\widetilde{1_x}\) is represented by the identity map near \(x\) and hence \(\widetilde{1_x}=\mathrm{1}_x\).
\item If \(g:x\to y\) and \(h:y\to z\), choose local sections \(\sigma_g:U\to\G\) and \(\sigma_h:V\to\G\) with \(x\in U\), \(y\in V\), \(s\circ\sigma_g=\mathrm{id}_U\), \(s\circ\sigma_h=\mathrm{id}_V\), \(\sigma_g(x)=g\), and \(\sigma_h(y)=h\). Shrinking \(U\) if necessary, we may assume \(\varphi_{(U,\sigma_g)}(U)\subseteq V\). On \(U\) the map \(\varphi_{(U,\sigma_g)}=r\circ\sigma_g\) represents \(\widetilde g\) and \(\varphi_{(V,\sigma_h)}=r\circ\sigma_h\) represents \(\widetilde h\). The product section \(U\to\G\), \(u\mapsto \sigma_h(\varphi_{(U,\sigma_g)}(u))\cdot\sigma_g(u)\), represents \(hg\), and its range map is the composition \(\varphi_{(V,\sigma_h)}\circ\varphi_{(U,\sigma_g)}\) near \(x\). Hence the germ of \(hg\) at \(x\) is the composite germ \(\widetilde h\circ\widetilde g\).
\end{enumerate}
\end{remark}

\begin{example}\label{ex:etale-groupoids}
\leavevmode
\begin{itemize}[noitemsep,nolistsep]
  \item \textbf{Spaces as \'{e}tale groupoids \cite[§1.3(1)]{crainic1999homology}.}
  Let \(X\) be a topological space. Consider the groupoid \((\G,\G_{0},r_{\G},s_{\G},\cdot_{\G},i_{\G})\) given by
  \(\G \coloneqq X\), \(\G_{0} \coloneqq X\),
  \(r_{\G} \coloneqq \mathrm{id}_{X}\), \(s_{\G} \coloneqq \mathrm{id}_{X}\),
  \(g^{-1}_{\G} \coloneqq g\), and \(g\cdot_{\G}h \coloneqq g\) whenever \(g,h\in X\) and \(g=h\). Then \((\G,\G_{0},r_{\G},s_{\G},\cdot_{\G},i_{\G})\) is a topological groupoid, and since \(r_{\G}\) and \(s_{\G}\) are homeomorphisms, \(\G\) is \'{e}tale.

  \item \textbf{Translation groupoid of a discrete group action \cite[§1.3(2)]{crainic1999homology}.}
  Let a discrete group \(\Gamma\) act on a locally compact Hausdorff space \(X\) by homeomorphisms via a left action \(\Gamma\times X\to X, (g,x)\mapsto g\cdot x\). The associated transformation groupoid \(\Gamma\ltimes X\) is the \'{e}tale groupoid with \( (\Gamma\ltimes X)_0\coloneqq X\), source and range maps \(s_{\Gamma\ltimes X}(g,x)\coloneqq x, r_{\Gamma\ltimes X}(g,x)\coloneqq g\cdot x\), unit map \(u_{\Gamma\ltimes X}(x)\coloneqq (e,x)\), inverse map \((g,x)^{-1}\coloneqq (g^{-1},g\cdot x)\), and multiplication \((h,y)\cdot(g,x)\coloneqq (hg,x)\) whenever \(s_{\Gamma\ltimes X}(h,y)=r_{\Gamma\ltimes X}(g,x)\), that is whenever \(y=g\cdot x\), so that explicitly \((h,g\cdot x)\cdot(g,x)=(hg,x)\).

  Since \(\Gamma\) is discrete, sets of the form \(\{g\}\times U\subseteq \Gamma\times X\) with \(U\subseteq X\) open form a basis. On \(\{g\}\times U\) the source map is the homeomorphism \((g,x)\mapsto x\), hence \(s_{\Gamma\ltimes X}\) is a local homeomorphism. Likewise, \(r_{\Gamma\ltimes X}\) restricts to the homeomorphism \((g,x)\mapsto g\cdot x\) from \(\{g\}\times U\) onto \(g\cdot U\). Therefore \(r_{\Gamma\ltimes X}\) is a local homeomorphism as well, and \(\Gamma\ltimes X\) is \'{e}tale.

  For a right action \((x,\gamma)\mapsto x\cdot\gamma\) one can also use the arrow space \(X\times\Gamma\) and define \(r_{X\rtimes\Gamma}(x,\gamma)\coloneqq x, s_{X\rtimes\Gamma}(x,\gamma)\coloneqq x\cdot\gamma\), with \(u_{X\rtimes\Gamma}(x)=(x,e)\), \((x,\gamma)^{-1}=(x\cdot\gamma,\gamma^{-1})\), and \((x,\gamma)\cdot(x\cdot\gamma,\eta)=(x,\gamma\eta)\).

  \item \textbf{Action groupoid of a right \(\G\)-space \cite[§1.3(6)]{crainic1999homology}.} Let \((\G,\G_0,r,s,m,(-)^{-1})\) be an \'{e}tale groupoid with unit map \(u: \G_0 \rightarrow \G, x \mapsto 1_x\). A right \(\G\)-space consists of a topological space \(X\), a continuous anchor map \(p:X\to\G_0\), and a continuous action map \(\cdot:X\ptimesr\G\to X\), \((x,g)\mapsto x\cdot g\), defined on the fibre product \(X\ptimesr\G=\{(x,g)\in X\times\G\mid p(x)=r(g)\}\), such that for all \((x,g)\in X\ptimesr\G\) one has \(p(x\cdot g)=s(g)\) and \(x\cdot 1_{p(x)}=x\), and whenever \((g,h)\in\G_2\) and \(p(x)=r(g)\) one has \((x\cdot g)\cdot h=x\cdot(gh)\).

  The action groupoid \(X\rtimes\G\) is the topological groupoid with objects \((X\rtimes\G)_0=X\) and arrows \((X\rtimes\G)_1=X\ptimesr\G\) with the subspace topology from \(X\times\G\), and structure maps \(r_{X\rtimes\G}(x,\gamma)=x\), \(s_{X\rtimes\G}(x,\gamma)=x\cdot\gamma\), \(u_{X\rtimes\G}(x)=(x,1_{p(x)})\), \((x,\gamma)^{-1}=(x\cdot\gamma,\gamma^{-1})\), and multiplication \((x,\gamma)\cdot(x\cdot\gamma,h)=(x,\gamma\eta)\), defined whenever \((\gamma,\eta)\in\G_2\). The space of composable pairs is \((X\rtimes\G)_2=\{((x,\gamma),(y,\eta))\in (X\rtimes\G)_1^2\mid y=x\cdot\gamma,\ (\gamma,\eta)\in\G_2\}\).

  Fix \((x,g)\in (X\rtimes\G)_1\). Choose an open bisection \(U\subseteq \G\) with \(g\in U\), so \(r|_U:U\to r(U)\) and \(s|_U:U\to s(U)\) are homeomorphisms onto open subsets of \(\G_0\). Choose an open neighbourhood \(W\subseteq X\) of \(x\) with \(p(W)\subseteq r(U)\). Then \(W\ptimesr U=\{(x',g')\in W\times U\mid p(x')=r(g')\}\) is an open neighbourhood of \((x,g)\) in \((X\rtimes\G)_1\).

  On \(W\ptimesr U\), the restriction of \(r_{X\rtimes\G}\) is a homeomorphism onto \(W\), with inverse \(W\to W\ptimesr U\), \(x'\mapsto \bigl(x',(r|_U)^{-1}(p(x'))\bigr)\). Moreover, the restriction of \(s_{X\rtimes\G}\) to \(W\ptimesr U\) is a homeomorphism onto its image. Indeed, for \(y\in s_{X\rtimes\G}(W\ptimesr U)\) we have \(p(y)\in s(U)\), hence \(k(y)\coloneqq (s|_U)^{-1}(p(y))\in U\) is well defined, and \(y\mapsto \bigl(y\cdot k(y)^{-1},k(y)\bigr)\) is an inverse to \(s_{X\rtimes\G}|_{W\ptimesr U}\) since \((y\cdot k(y)^{-1})\cdot k(y)=y\) and \(p(y\cdot k(y)^{-1})=r(k(y))\). Therefore \(r_{X\rtimes\G}\) and \(s_{X\rtimes\G}\) are local homeomorphisms, and \(X\rtimes\G\) is \'{e}tale.  
\end{itemize}
\end{example}

\begin{definition}[{\cite[§1.4]{crainic1999homology}}]
Let \((\K,\K_{0},r_{\K},s_{\K},\cdot_{\K},i_{\K})\) and \((\G,\G_{0},r_{\G},s_{\G},\cdot_{\G},i_{\G})\) be \'{e}tale groupoids.
A homomorphism of \'{e}tale groupoids \(\varphi:\K\to\G\) is a pair of continuous maps \(\varphi_{0}:\K_{0}\to\G_{0}\) and \(\varphi_{1}:\K\to\G\)
such that the following identities hold:
\begin{align*}
&r_{\G}\circ \varphi_{1} = \varphi_{0}\circ r_{\K}, & \ &s_{\G}\circ \varphi_{1} = \varphi_{0}\circ s_{\K}, \\
&\varphi_{1}\circ (i_{\K}) = (i_{\G})\circ \varphi_{1},  & &\varphi_{1}\circ u_{\K} = u_{\G}\circ \varphi_{0}, \\
&\varphi_{1}(g\cdot_{\K} h) = \varphi_{1}(g)\cdot_{\G} \varphi_{1}(h) &
&\text{for all }(g,h) \in \K\times_{\K_{0}}\K.
\end{align*}
This is precisely the data of a functor of small categories \(\K\to\G\).
\end{definition}

Isomorphism of \'{e}tale groupoids is usually too rigid for geometric purposes, since different groupoids can present the same quotient or orbit data. Morita equivalence isolates the underlying geometric object by requiring a homomorphism \(\varphi:\K\to\G\) to behave like an equivalence of categories in a way compatible with the topology. Recall that a functor between small categories is an equivalence if and only if it is essentially surjective on objects and fully faithful on morphisms. In the \'{e}tale setting these two conditions are reformulated in terms of local homeomorphisms of the unit spaces and of the bibundle of arrows, leading to the notion of Morita equivalence for \'{e}tale groupoids.

\begin{definition}[Morita equivalence {\cite[§1.5]{crainic1999homology}}]
\label{def:morita}
Let \(\K\) and \(\G\) be \'{e}tale groupoids and let \(\varphi:\K\to\G\) be a homomorphism with components \(\varphi_0:\K_0\to\G_0\) and \(\varphi_1:\K_1\to\G_1\). We say that \(\varphi\) is a Morita equivalence, also known as a weak or essential equivalence, if:
\begin{enumerate}[noitemsep,nolistsep]
\item \textbf{Essential surjectivity (\'{e}tale surjection).}
Consider the fibre product
\[
\K_0 \phitimesr \G_1
=
\{ (y,g)\in \K_0\times \G_1 \mid \varphi_0(y)=r_{\G}(g) \},
\]
with projections \(\pi_1(y,g)=y\) and \(\pi_2(y,g)=g\).
The map
\[
s_{\G}\circ\pi_2:\K_0 \phitimesr \G_1 \to \G_0,
\qquad
(y,g)\mapsto s_{\G}(g),
\]
is a surjective local homeomorphism.

\item \textbf{Full faithfulness (pullback square).}
The canonical square
\[
\begin{tikzcd}
  \K_1 \arrow[r,"{\varphi_1}"] \arrow[d,"{(r_{\K},s_{\K})}"'] & \G_1 \arrow[d,"{(r_{\G},s_{\G})}"] \\
  \K_0\times \K_0 \arrow[r,"{\varphi_0\times\varphi_0}"'] & \G_0 \times \G_0
\end{tikzcd}
\]
is a pullback in the sense that the natural map
\[
\K_1 \to (\K_0\times\K_0)\phiphitimesrr\G_1,
\qquad
k \mapsto \bigl((r_{\K}(k),s_{\K}(k)),\varphi_1(k)\bigr),
\]
is a homeomorphism, where the fibre product is
\end{enumerate}
\[
(\K_0\times\K_0)\phiphitimesrr\G_1
=
\bigl\{((y_1,y_2),g)\in(\K_0\times\K_0)\times\G_1
\mid (\varphi_0(y_1),\varphi_0(y_2))=(r_{\G}(g),s_{\G}(g))\bigr\}.
\]
\end{definition}

\begin{remark}
The essential surjectivity condition says that every object of \(\G\) is locally in the image of \(s_{\G}\circ\pi_2\), and hence admits local lifts along \(s_{\G}\circ\pi_2\) after restricting to suitable neighbourhoods in \(\G_0\). The pullback condition identifies \(\K_1\) with the space of arrows of \(\G\) between points in the image of \(\K_0\), so that arrows of \(\G\) over \(\varphi_0\) correspond uniquely and continuously to arrows of \(\K\). Together these conditions express that \(\varphi\) is fully faithful and essentially surjective in a way compatible with the topologies on \(\K\) and \(\G\).
\end{remark}

When these conditions hold we write \(\varphi:\K \xrightarrow{\sim} \G\). Two \'{e}tale groupoids \(\Hh\) and \(\G\) are Morita equivalent if there exist Morita equivalences \(\Hh \xleftarrow{\sim} \K \xrightarrow{\sim} \G\). This generates an equivalence relation, and one often works in the localisation of the category of \'{e}tale groupoids obtained by formally inverting all Morita equivalences. A morphism \(\Hh\to\G\) there can be represented by a zig--zag \(\Hh \xleftarrow{\sim} \K \xrightarrow{\sim} \G\) \cite[§1.5]{crainic1999homology}.

\begin{definition}[{\cite[§2.1]{matui2012homology}}]
For an \'{e}tale groupoid \(\G\) and \(Y\subseteq \G_0\) the reduction is
\(
\G\vert_Y \coloneqq r^{-1}(Y)\cap s^{-1}(Y),
\)
equipped with the subspace topology. It is an \'{e}tale subgroupoid with unit space \(Y\) whenever \(Y\) is open.
\end{definition}

\begin{definition}[{\cite[§2.1]{matui2012homology}}]
Let \((\G,\G_{0},r_{\G},s_{\G},\cdot_{\G},i_{\G})\) be an \'{e}tale groupoid and write \(r\coloneqq r_{\G}\), \(s\coloneqq s_{\G}\).
A subset \(U\subseteq \G\) is a \(\G\)\nobreakdash-bisection if the restrictions \(r|_U:U\to r(U)\) and \(s|_U:U\to s(U)\) are injective. If in addition \(U\) is open, we call it an open \(\G\)\nobreakdash-bisection.
For subsets \(U,V\subseteq\G\) set
\[
U^{-1}\coloneqq \{g\in\G \mid g^{-1}\in U\},
\qquad
UV\coloneqq \{gh \mid g\in U,\ h\in V,\ s(g)=r(h)\}.
\]
If \(U\) and \(V\) are \(\G\)\nobreakdash-bisections, then \(U^{-1}\) and \(UV\) are again \(\G\)\nobreakdash-bisections. If \(U\) and \(V\) are open \(\G\)\nobreakdash-bisections, then \(U^{-1}\) and \(UV\) are open \(\G\)\nobreakdash-bisections.
\end{definition}

\begin{remark}
In the \'{e}tale case every arrow \(g\in\G\) admits an open neighbourhood \(U\subseteq\G\) which is a \(\G\)\nobreakdash-bisection.
On such a set, the restrictions \(r|_U:U\to r(U)\) and \(s|_U:U\to s(U)\) are homeomorphisms onto open subsets of \(\G_0\) \cite[§2.4]{Sims2018}.
\end{remark}

\begin{lemma}\label{lem:discrete-units-implies-discrete-nerve}
Let \(\G\) be an \'{e}tale groupoid. If the unit space \(\G_0\) is discrete, then \(\G\) is discrete, and for every \(n\ge 0\) the space of \(n\)-composables \(\G_n\) is discrete.
\end{lemma}

\begin{proof}
Assume \(\G_0\) is discrete.

\begin{itemize}[noitemsep,nolistsep]
\item \textbf{\(\G\) is discrete.}
Since \(\G\) is \'{e}tale, the range map \(r:\G\to\G_0\) is a local homeomorphism.
Fix \(g\in\G\) and set \(u\coloneqq r(g)\in\G_0\).
Because \(\G_0\) is discrete, the singleton \(\{u\}\) is open.
As \(r\) is a local homeomorphism, there exists an open neighbourhood \(U\subset\G\) of \(g\) such that \(r|_U:U\to r(U)\) is a homeomorphism onto an open subset \(r(U)\subset\G_0\).
Then \(r(U)\cap\{u\}\) is open in \(r(U)\), hence
\(
U'\coloneqq (r|_U)^{-1}\bigl(r(U)\cap\{u\}\bigr)
\)
is open in \(U\), hence open in \(\G\).
Moreover \(g\in U'\) and \(r(U')=\{u\}\).
Since \(r|_U\) is injective, \(U'\) contains at most one point, hence \(U'=\{g\}\).
Thus \(\{g\}\) is open in \(\G\), so \(\G\) is discrete.

\item \textbf{\(\G^n\) is discrete for all \(n\ge 0\).}
For \(n=0\) we have \(\G_0\) discrete by assumption.
For \(n\ge 1\), the product \(\G^n\) is discrete because \(\G\) is discrete.
\end{itemize}
The space \(\G_n\) of composable \(n\)-tuples is a subspace of \(\G^n\), hence discrete as well.
\end{proof}

\begin{definition}
A subset \(X\subseteq \G_0\) is called full if \(r\bigl(s^{-1}(X)\bigr)=\G_0\).
\end{definition}

\begin{definition}[Kakutani equivalence {\cite[Definition~3.8]{matui2022long}}]\label{def:kakutani-equivalence}
Let \(\G\) and \(\Hh\) be \'{e}tale groupoids.
We say that \(\G\) and \(\Hh\) are Kakutani equivalent if there exist full clopen subsets
\(X\subseteq \G_0\) and \(Y\subseteq \Hh_0\) such that the reductions
\(\G|_X\) and \(\Hh|_Y\) are isomorphic as \'{e}tale groupoids.
\end{definition}

\begin{remark}
Let \(\G\) be an \'{e}tale groupoid and let \(X\subseteq \G_0\) be clopen and full. The reduction \(\G|_X\) has unit space \(X\) and arrow space \(\G|_X \coloneqq \{g\in\G \mid r_{\G}(g)\in X,\ s_{\G}(g)\in X\}\). Since \(X\) is clopen, the subsets \(r_{\G}^{-1}(X)\) and \(s_{\G}^{-1}(X)\) are open in \(\G\), and hence \(\G|_X = r_{\G}^{-1}(X)\cap s_{\G}^{-1}(X)\) is open in \(\G\). The structure maps of \(\G|_X\) are the restrictions of those of \(\G\):
\[
r_{\G|_X} = r_{\G}|_{\G|_X},\quad
s_{\G|_X} = s_{\G}|_{\G|_X},\quad
(g,h)\mapsto g\cdot_{\G}h,\quad
g\mapsto g^{-1},
\]
with composition defined whenever \(s_{\G}(g)=r_{\G}(h)\in X\). Since \(r_{\G}\) and \(s_{\G}\) are local homeomorphisms, their restrictions \(r_{\G|_X}\) and \(s_{\G|_X}\) are local homeomorphisms as well. Thus \(\G|_X\) is again an \'{e}tale groupoid with unit space \(X\). Fullness of \(X\), that is \(r_{\G}\bigl(s_{\G}^{-1}(X)\bigr)=\G_0\), means that every unit \(x\in\G_0\) is the range of some arrow whose source lies in \(X\). This means that every \(\G\)\nobreakdash-orbit meets \(X\). Hence the reduction \(\G|_X\) still sees the entire orbit structure of \(\G\).
If \(\G|_X\cong\Hh|_Y\) for full clopen \(X\subseteq\G_0\) and \(Y\subseteq\Hh_0\), then \(\G\) and \(\Hh\) have the same dynamics up to cutting down to representatives in \(X\) and \(Y\), which is what Kakutani equivalence is designed to capture.
\end{remark}

\begin{lemma}[{\cite[Lemma~2.4.9]{Sims2018}}]
\label{lem:bisection-basis}
If \(\G\) is second countable, Hausdorff, and \'{e}tale, then \(\G_1\) admits a countable base consisting of open bisections.
\end{lemma}

\begin{proof}
Let \(\mathcal{B}\) be a countable base for the topology on \(\G_1\). Fix \(B\in\mathcal{B}\). For every \(\gamma\in B\), since \(\G\) is \'{e}tale there exists an open bisection \(U_\gamma\subseteq \G_1\) with \(\gamma\in U_\gamma\subseteq B\). Thus \(\{U_\gamma\}_{\gamma\in B}\) is an open cover of \(B\) by open bisections. Since \(\G_1\) is second countable, it is Lindelöf, so there exists a countable subcover \(\{U_{B,n}\}_{n\in\mathbb{N}}\) of \(B\) by open bisections. Define
\[
\mathcal{W}\coloneqq \{U_{B,n}\mid B\in\mathcal{B},\ n\in\mathbb{N}\}.
\]
Then \(\mathcal{W}\) is countable. To see that \(\mathcal{W}\) is a base, let \(W\subseteq\G_1\) be open and \(\gamma\in W\). Choose \(B\in\mathcal{B}\) with \(\gamma\in B\subseteq W\). Since \(\{U_{B,n}\}_{n\in \mathbb{N}}\) covers \(B\), there exists \(n\) with \(\gamma\in U_{B,n}\subseteq B\subseteq W\). Hence \(\mathcal{W}\) is a countable base consisting of open bisections.
\end{proof}

\begin{corollary}[{\cite[Corollary~2.4.10]{Sims2018}}]
\label{cor:discrete}
If \(\G\) is \'{e}tale, then for every \(x\in \G_0\) the sets \(r^{-1}(x)\), \(s^{-1}(x)\), and \(\G_x \coloneqq \{\gamma\in\G \mid r(\gamma)=s(\gamma)=x\}\) are discrete in the subspace topology.
\end{corollary}

\begin{proof}
Fix \(x\in\G_0\) and \(\gamma\in r^{-1}(x)\). Since \(\G\) is \'{e}tale, there exists an open bisection \(U\subseteq\G\) with \(\gamma\in U\). On \(U\) the restriction \(r|_U:U\to r(U)\) is injective, so \(U\cap r^{-1}(x)\) contains at most one point. Since \(r(\gamma)=x\), we have \(U\cap r^{-1}(x) = \{\gamma\}\), and \(\{\gamma\}\) is open in the subspace \(r^{-1}(x)\). Thus \(r^{-1}(x)\) is discrete. The same argument applied to \(s\) shows that \(s^{-1}(x)\) is discrete. Now let \(\gamma\in\G_x = r^{-1}(x)\cap s^{-1}(x)\). From the first two steps there exist open sets \(U_r,U_s\subseteq\G\) such that \(U_r\cap r^{-1}(x) = \{\gamma\}\) and \(U_s\cap s^{-1}(x) = \{\gamma\}\). Then \((U_r\cap U_s)\cap \G_x = (U_r\cap r^{-1}(x))\cap (U_s\cap s^{-1}(x)) = \{\gamma\}\), so \(\{\gamma\}\) is open in the subspace \(\G_x\). Therefore \(\G_x\) is discrete.
\end{proof}

\begin{lemma}[{\cite[Lemma~2.4.11]{Sims2018}}]
\label{lem:mult-open}
Let \(\G\) be a topological groupoid. If the range map \(r:\G\to\G_{0}\) is open, then the multiplication \(m:\G_{2}\to\G\) is an open map. In particular, \(m\) is open for any \'{e}tale groupoid.
\end{lemma}

\begin{proof}
Let \(U,V\subseteq\G\) be open and let \((\alpha,\beta)\in\G_{2}\cap(U\times V)\), so \(\alpha\in U\), \(\beta\in V\) and the product \(\alpha\beta\) is defined. We show that \(\alpha\beta\) is an interior point of
\[
UV \coloneqq \{\mu\nu \mid \mu\in U,\ \nu\in V,\ s(\mu)=r(\nu)\}
=
m\bigl((U\times V)\cap\G_{2}\bigr).
\]
Fix a decreasing neighbourhood base \((U_{j})_{j\in J}\) of \(\alpha\) in \(\G\), with each \(U_{j}\subseteq U\). Since \(r\) is open, each \(r(U_{j})\) is an open neighbourhood of \(r(\alpha)=r(\alpha\beta)\) in \(\G_{0}\). Let \((\gamma_{i})_{i\in I}\) be a net in \(\G\) with \(\gamma_{i}\to\alpha\beta\). Then \(r(\gamma_{i})\to r(\alpha\beta)=r(\alpha)\).
For each \(j\in J\), the set \(r(U_{j})\) is an open neighbourhood of \(r(\alpha)\), so there exists \(i_{0}(j)\in I\) such that \(r(\gamma_{i})\in r(U_{j})\) for all \(i\ge i_{0}(j)\). For each pair \((i,j)\) with \(i\ge i_{0}(j)\), choose \(\alpha_{(i,j)}\in U_{j}\) such that \(r(\alpha_{(i,j)})=r(\gamma_{i})\).
Let \(D\coloneqq \{(i,j)\in I\times J\mid i\ge i_{0}(j)\}\), directed by \((i,j)\le (i',j')\) if \(i\le i'\) and \(j\le j'\). Then \(\alpha_{(i,j)}\to \alpha\) as \((i,j)\) tends to infinity in \(D\). Moreover, inversion is continuous, hence \(\alpha_{(i,j)}^{-1}\to\alpha^{-1}\).
For each \((i,j)\in D\) the pair \((\alpha_{(i,j)}^{-1},\gamma_{i})\) lies in \(\G_{2}\) because
\(
s(\alpha_{(i,j)}^{-1})=r(\alpha_{(i,j)})=r(\gamma_{i}).
\)
By continuity of multiplication,
\[
\alpha_{(i,j)}^{-1}\gamma_{i} \to \alpha^{-1}(\alpha\beta)=\beta
\]
in \(\G\) along the net indexed by \(D\). Since \(V\) is an open neighbourhood of \(\beta\), there exists \((i_{1},j_{1})\in D\) such that \(\alpha_{(i,j)}^{-1}\gamma_{i}\in V\) for all \((i,j)\ge (i_{1},j_{1})\). For such \((i,j)\) we have \(\alpha_{(i,j)}\in U\), \(\alpha_{(i,j)}^{-1}\gamma_{i}\in V\), and the pair \((\alpha_{(i,j)},\alpha_{(i,j)}^{-1}\gamma_{i})\) is composable. Therefore
\(
\gamma_{i}=\alpha_{(i,j)}\bigl(\alpha_{(i,j)}^{-1}\gamma_{i}\bigr)\in UV
\)
for all sufficiently large \((i,j)\). Since \((\gamma_{i})\) was an arbitrary net converging to \(\alpha\beta\), this shows that \(\alpha\beta\) lies in the interior of \(UV\). Hence \(UV\) is open and \(m\) is an open map. If \(\G\) is \'{e}tale, then \(r\) is a local homeomorphism, hence an open map, so \(m\) is open.
\end{proof}

Principal \(\G\)\nobreakdash-bundles enter naturally when one wants to express Morita equivalence in geometric rather than purely functorial terms. In our setting, Morita equivalent \'{e}tale groupoids are related by principal bibundles, and these bibundles are built from right and left principal \(\G\)\nobreakdash-bundles as in the definition below. Introducing principal bundles therefore prepares the ground for describing Morita equivalence via correspondences, and for proving that invariants such as Moore homology are preserved under these geometric equivalences of groupoids.

\begin{definition}[Principal \({\G}\)-bundle {\cite[§1.6]{crainic1999homology}}]
Let \(X\) be a topological space and \(\G\) an \'{e}tale groupoid.
A right principal \(\G\)-bundle over \(X\) consists of a topological space \(P\), a surjective open map \(\pi:P\to X\), an anchor \(\epsilon:P\to \G_0\), and a continuous right action \(P\stimesr\G_1 \to P, (p,g)\mapsto p\cdot g,\) defined when \(\epsilon(p)=r(g)\), such that \(\epsilon(p\cdot g)=s(g), p\cdot 1_{\epsilon(p)}=p, (p\cdot g)\cdot h=p\cdot(gh)\), and the canonical map of fibre products \(P\stimesr\G_1 \to P\pitimespi P, (p,g)\mapsto (p, p\cdot g)\), is a homeomorphism.
\end{definition}

In the context of Morita equivalence we do not only need right principal \(\G\)\nobreakdash-bundles, but bundles that carry in addition a compatible action of a second groupoid \(\K\). A \(\K\)\nobreakdash-equivariant principal \(\G\)\nobreakdash-bundle is precisely a principal \(\G\)\nobreakdash-bundle together with a left \(\K\)\nobreakdash-action which commutes with the right \(\G\)\nobreakdash-action and respects the projection to the base. Such objects organise into \((\K,\G)\)\nobreakdash-bibundles and give the geometric realisation of Morita equivalences between \(\K\) and \(\G\). In particular, every Morita equivalence can be encoded by a \(\K\)\nobreakdash-equivariant principal \(\G\)\nobreakdash-bundle, and conversely these bibundles provide the correct framework to transport structures such as homology and cohomology functorially along equivalences of \'{e}tale groupoids.

\begin{definition}[{\cite[§1.6]{crainic1999homology}}]
Let \(\K\) and \(\G\) be \'{e}tale groupoids.

A \(\K\)\nobreakdash-equivariant principal \(\G\)\nobreakdash-bundle consists of:
\begin{itemize}[nosep]
  \item a topological space \(P\) together with continuous maps \(\pi:P\to \K_0, \varepsilon:P\to \G_0\),
  \item a continuous right \(\G\)\nobreakdash-action with anchor \(\varepsilon\), \(P\epsilontimesr\G_1 \to P, (p,g)\mapsto p\cdot g\), defined whenever \(\varepsilon(p)=r_{\G}(g)\), such that \(\varepsilon(p\cdot g)=s_{\G}(g), p\cdot 1_{\varepsilon(p)}=p, (p\cdot g)\cdot h=p\cdot(gh)\) for all composable \(g,h\in\G_1\), and the canonical map of fibre products \(\Theta:\ P\stimesr\G_1 \to P\pitimespi P, (p,g)\mapsto \bigl(p, p\cdot g\bigr)\), is a homeomorphism. In particular, \(\pi:P\to\K_0\) is a surjective open map and \((P,\pi,\varepsilon)\) is a principal right \(\G\)\nobreakdash-bundle over \(\K_0\) in the sense defined above,
  \item a continuous left \(\K\)\nobreakdash-action with anchor \(\pi\), \(\K_1\pitimespi P \to P, (k,p)\mapsto k\cdot p\), defined whenever \(s_{\K}(k)=\pi(p)\), such that \(\pi(k\cdot p)=r_{\K}(k), 1_{\pi(p)}\cdot p=p, (k\ell)\cdot p=k\cdot(\ell\cdot p)\) for all composable \(k,\ell\in\K_1\), and \(\varepsilon(k\cdot p)=\varepsilon(p)\) for all \((k,p)\in\K_1\pitimespi P\),
  \item the left \(\K\)\nobreakdash-action and the right \(\G\)\nobreakdash-action commute, \((k\cdot p)\cdot g = k\cdot(p\cdot g)\) whenever defined.
\end{itemize}
\end{definition}

This says that the left \(\K\)\nobreakdash-action is by bundle automorphisms of the principal right \(\G\)\nobreakdash-bundle \((P,\pi,\varepsilon)\): the map \(\Theta\) is \(\K\)\nobreakdash-equivariant for the diagonal \(\K\)\nobreakdash-action on \(P\pitimespi P\) and the induced \(\K\)\nobreakdash-action on \(P\stimesr\G_1\). A morphism of \(\K\)\nobreakdash-equivariant principal \(\G\)\nobreakdash-bundles is a homeomorphism \(f:P\to P'\) such that \(\pi'=\pi\circ f, \varepsilon'=\varepsilon\circ f, f(k\cdot p)=k\cdot f(p)\) and \(f(p\cdot g)=f(p)\cdot g\). Isomorphism classes of such objects are the generalised morphisms \(\K\dashrightarrow \G\) \cite[§1.6]{crainic1999homology}. To apply the homology theories of \cite{crainic2000homology,matui2012homology,matui2022long} we adopt the setting of \cite[§1.8]{crainic1999homology}. Throughout, all groupoids are \'{e}tale and their unit and morphism spaces are Hausdorff, locally compact and second countable. Under these standing assumptions the nerves \(\G_n\) inherit compatible topologies, the bisection bases of Lemma~\ref{lem:bisection-basis} exist, the fibres in Corollary~\ref{cor:discrete} are discrete and the multiplication is open by Lemma~\ref{lem:mult-open}. These properties enter both the construction of convolution algebras \cite{Sims2018} and the homology long exact sequences \cite{matui2022long}.