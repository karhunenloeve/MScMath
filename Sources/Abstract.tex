\begin{center}
  \textsc{Abstract}
\end{center}
\noindent

We study homology of ample groupoids through the compactly supported Moore chain complex of the nerve \(\G_\bullet\). Throughout we work under standing hypotheses ensuring that the pushforward maps along the face maps are defined on compactly supported chains (in particular, \'etaleness so that the face maps are local homeomorphisms, and local compactness/Hausdorffness on the relevant spaces so that compact supports behave as expected). For a topological abelian group \(A\) we define, for each \(n\ge 0\), the Moore chain group \(C_c(\G_n,A)\) of compactly supported continuous \(A\)\nobreakdash-valued functions on \(\G_n\), with boundary
\(\partial_n^A=\sum_{i=0}^n(-1)^i(d_i)_*\)
induced by the face maps. This yields homology groups \(H_n(\G;A)\) which are functorial for continuous \'etale homomorphisms and compatible with the standard operations used in computations (reduction to saturated clopen subsets and, in the ample setting, invariance under Kakutani equivalence). We reprove Matui’s long exact sequences in a Moore--complex formulation and identify the comparison maps at the chain level (Theorem~\ref{thm:LES-regular-diagram}).

A main theme is a universal coefficient phenomenon for Moore homology with compact supports. For discrete abelian \(A\) we prove a natural short exact sequence
\[
0
\to H_n(\G)\otimes_{\mathbb Z} A
\xrightarrow{\ \iota_n^{\G}\ }
H_n(\G;A)
\xrightarrow{\ \kappa_n^{\G}\ }
\operatorname{Tor}_1^{\mathbb Z}\bigl(H_{n-1}(\G),A\bigr)
\to 0,
\]
natural in both \(\G\) and \(A\) (Theorem~\ref{thm:UCT-homology-G}). The key input is the chain-level identification
\(
C_c(\G_n,\mathbb Z)\otimes_{\mathbb Z} A \cong C_c(\G_n,A),
\)
which reduces the groupoid statement to the classical algebraic UCT applied to the free chain complex \(C_c(\G_\bullet,\mathbb Z)\).
We then isolate a sharp obstruction to extending this mechanism beyond discrete coefficients: for a locally compact totally disconnected Hausdorff space \(X\) with a basis of compact open sets and a topological abelian group \(A\), the image of the canonical comparison map
\(\Phi_X:C_c(X,\mathbb Z)\otimes_{\mathbb Z}A\to C_c(X,A)\)
consists exactly of the compactly supported functions with finite image. Consequently, \(\Phi_X\) is surjective (equivalently, an isomorphism) if and only if every \(f\in C_c(X,A)\) has finite image (Corollary~\ref{cor:discrete-necessity}). In particular, discreteness of \(A\) is sufficient for \(\Phi_X\) to be an isomorphism; however, the converse can fail in general. Under additional mild countability hypotheses on \(A\) one can nonetheless construct, for suitable non-discrete ample spaces \(X\), compactly supported continuous functions \(X\to A\) with infinite image, forcing \(\Phi_X\) to fail to be surjective; this explains why the classical UCT is, in the Moore framework, essentially a discrete-coefficient phenomenon.

Finally, we develop a Mayer--Vietoris principle for ample groupoids and topological coefficients. For a topological abelian group \(A\) and a clopen saturated cover \(\G_0=U_1\cup U_2\), we construct a short exact sequence of Moore chain complexes and derive the Moore--Mayer--Vietoris long exact homology sequence (Theorem~\ref{thm:MV-long-exact})
\[
\begin{tikzcd}[arrow style=math font,cells={nodes={text height=2ex,text depth=0.75ex}}]
\cdots
& H_{n-1}(\mathcal{G}|_{U_1};A)\oplus H_{n-1}(\mathcal{G}|_{U_2};A)
  \arrow[l]
  \arrow[draw=none]{d}[name=Y,shape=coordinate]{}
& H_{n-1}(\mathcal{G}|_{U_1\cap U_2};A)
  \arrow[l,"{H_{n-1}(\alpha_\bullet)}"']
\\
H_{n}(\mathcal{G};A)
  \arrow[curarrow={Y}{\partial_{n}}]{urr}
& H_{n}(\mathcal{G}|_{U_1};A)\oplus H_{n}(\mathcal{G}|_{U_2};A)
  \arrow[l,"{H_n(\beta_\bullet)}"']
  \arrow[draw=none]{d}[name=Z,shape=coordinate]{}
& H_{n}(\mathcal{G}|_{U_1\cap U_2};A)
  \arrow[l,"{H_n(\alpha_\bullet)}"']
\\
H_{n+1}(\mathcal{G};A)
  \arrow[curarrow={Z}{\partial_{n+1}}]{urr}
& H_{n+1}(\mathcal{G}|_{U_1};A)\oplus H_{n+1}(\mathcal{G}|_{U_2};A)
  \arrow[l,"{H_{n+1}(\beta_\bullet)}"']
& \cdots \arrow[l]
\end{tikzcd}
\]
which is designed for explicit computations by cutting the unit space into saturated clopen pieces and recovering \(H_\bullet(\G;A)\) from the corresponding reductions. Combined with the UCT in the discrete-coefficient regime, this framework isolates how torsion in integral homology contributes additional homology via \(\operatorname{Tor}_1^{\mathbb Z}\), a phenomenon illustrated concretely in a later example built from standard ample groupoids such as those associated to shifts of finite type.